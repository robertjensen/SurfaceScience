\newpage
\chapter{The XPS Method}

          X-ray induced Photoelectron Spectroscopy (XPS) relies on a
          well know principle namely the photo-electric effect, which
          was explained by Einstein in 1905. When a surface is bombarded
          with photons with sufficient energy an electron may adsorb
          all the energy of the photon and be able to escape the solid
          with a kinetic energy reflecting the photon energy as well
          as the bonding energy of the electron. It was not until in
          the late sixties that this process was brought into
          operation by Kai Siegbahn who later received the Nobel price
          for his work. The process is illustrated in Figure 4.1.
          \vspace*{9.5cm}

            \noindent {\bf Figure 4.1} Schematics of the photoemission
        process.\\

           X-rays are generated with a suitable  energy
          so that the electrons, excited from the atoms,
          have a kinetic  energy  in  the  range  where  high  surface
          sensitivity can be  obtained.  The  kinetic  energy  of  the
          emitted electron relative to the vacuum level will be  given
          by \begin{equation}  E_{kin}=h\nu-E_{B}-\Phi  \end{equation}
          where h$\nu$ is the photon energy of the X-ray,  E$_{B}$  is
          the binding energy of the electron in the solid referred  to
          the Fermi level, and $\Phi$ is the  work  function  for  the
          solid. It is obvious that if we can generate a monochromatic
          source of X-rays it is possible  to  map  out  the  electron
          density as a function of binding  energy  by  measuring  the
          kinetic energy of the emitted electrons. Since the  electron
          density as a function of binding  energy  is  characteristic
          for each element this is a useful method  to  determine  not
          only which elements are present in the surface but also  how
          much there is  of each. Furthermore,  small  shifts  in  the
          measured binding energy  reflect  changes  of  the  chemical
          state of the elements. The XPS  method  has  therefore  also
          been called  Electron  Spectroscopy  for  Chemical  Analysis
          (ESCA).


             \section{X-ray Sources}

             When a material is bombarded with electrons there will, as
          we saw in chapter 2, be possibilities for ionisation of the
          atoms in the material. The exited atom will after a very
          short time relax by a process where a weakly bound electron
          is dropping down into the created vacancy. The energy
          released by this process can now be used to either excite
          another bound electron out of the atom (an Auger process) or
          in this case more interesting to emit a photon. In Figure
          4.2 is shown a dual anode which is bombarded with 15 kV
          electrons.

\vspace{8cm}

             \noindent {\bf Figure 4.2} Schematics of a commercial dual
          X-ray anode.\\



           A thin aluminium foil is mounted just in front of
          the anode in order to prevent stay electrons and outgassing
          to the rest of the chamber.


           The process is shown for an aluminium atom  in
          Figure 4.3. Initially a hole is formed in the 1s level.  The
          atom may now relax by either letting an  electron  from  the
          2p$_{\frac{3}{2}}$ or 2p$_{\frac{1}{2}}$ fall down and  emit
          a photon. The energy splitting between the two  p  levels  is
          due to spin-orbit coupling and is very small (0.4-0.5 eV).\\



\vspace{14cm}

       \noindent     {\bf Figure 4.3} Schematics of the transitions in
          aluminium leading to the Al$_{K\alpha}$ radiation.\\

          Unfortunately, the short lifetime of the excited states
          involved will also influence the widths of the emitted
          X-rays. The relations are given through the Heisenberg
          uncertainty relation for energy and time \begin{equation}
          \Delta E \Delta t \geq \frac{h}{2\pi} \end{equation} and
          leads to, as shown in Figure 4.4, a broadening of 0.7 eV
          for each of the two lines \cite{siegbahn1}. Thus the X-ray emitted by this
          transition will effectively be one line which will be roughly
          1.0 eV broad at Full Width Half Maximum (FWHM) with an
          energy of 1486.6 eV.\\

\vspace{1cm}

  \noindent  {\bf  Figure 4.4}  The  line width   of\\   the
  K$_{\alpha 1}$ and K$_{\alpha2}$ resulting\\ in the K$_{\alpha
 12}$ line \cite{siegbahn1}.\\

                                              \vspace{5cm}

             Due to an old X-ray notation this transition is called an
          Al$_{K_{\alpha 12}}$ line indicating that a hole is
          generated in the K shell (main quantum number one i.e. 1s).
          A quite similar transition is found in magnesium which will
          be a bit more narrow (0.7 eV) and has an energy of 1253.6
          eV. Many other materials may be used as anodes as shown in
          table 4.1, but in general Mg and Al are the preferred
          species to have on a dual anode as shown in Figure 4.2.\\


{\bf Table 4.1}\\


             \begin{tabular}{||l|l|l||} \hline
          X-ray line & Energy (eV) & FWHM (eV) \\ \hline
          Mg$_{K_{\alpha}}$ & 1253.6 & 0.70 \\ \hline
          Al$_{K_{\alpha}}$ & 1486.6 & 0.85 \\ \hline
          Si$_{K_{\alpha}}$ & 1739.5 & 1.2 \\ \hline
          Zr$_{L_{\alpha}}$ & 2042.4 & 0.77 \\ \hline
          Ag$_{L_{\alpha}}$ & 2984 & $<$ 3 \\ \hline
          Ti$_{L_{\alpha}}$ & 4511 & 1.4 \\ \hline
          Cr$_{L_{\alpha}}$ & 5414.7 & 1.8 \\ \hline
          Cu$_{L_{\alpha}}$ & 8048 & 2.5 \\ \hline
          \end{tabular}

                   \vspace{1cm}

             The reason being that a narrow line is desired in
          combination with sufficient energy that many core levels in
          the sample material can be reached.



             Usually the Al$_{K_{\alpha 12}}$ and Mg$_{K_{\alpha 12}}$
          are sufficient for most purposes, but when detailed studies
          are necessary it is important to consider the quality of the
          X-ray source as there will be other possibilities for
          transitions in the anode. Figure 4.5 shows schematically the
          number of photons emitted from an aluminium anode as a
          function of photon energy when it is bombarded with E$_{p}$
          = 15 kV electrons.\\

\vspace{16cm}

         \noindent   {\bf Figure 4.5} Schematics of the X-ray emission
        spectrum for an anode bombarded with high energy (E$_{p}$)
       electrons.\\


             There will be a continuously weak background of photons
          due to the deceleration of the electrons in the anode.
          Besides this there is a strong and a discrete contribution
          from the 1s-2p transition as discussed above. However, other
          transitions can also be observed. This is illustrated in
          Figure 4.6 where the number of photons from a magnesium
          anode is plotted against energy measured relative to the
          Mg$_{K_{\alpha 12}}$ line. The $\alpha_{34}$ lines are due
          to a double ionisation of the magnesium atom and are located
          approximately at 10 eV higher energy \cite{krause}. The
          $\beta$ line is due to a transition where  a valence
          electron which falls down and occupies a hole in the
          1s shell. Usually the $\beta$ is so weak that it seldom may
          be noticed.\\

\vspace{12cm}


\noindent     {\bf Figure 4.6} The relative intensity of X-rays emitted
          from a magnesium anode plotted relative to the K$_{\alpha
          12}$ energy. Notice that the intensity is on a logarithmic
          scale \cite{krause}.\\

             Shown in table 4.2 are the X-ray satellite lines of
          magnesium and aluminium positioned relative to the main lines.
          Notice that more than 10-15\% of the X-ray intensity will be
          in the different satellite structures. All these various
          problems with different X-ray lines can be avoided if we let
          the X-ray pass a monochromator. This is easily established
          by diffraction of the X-rays in a Si single crystal. The
          only major drawback by this procedure is the extensive loss
          of sensitivity (actually photon intensity) and naturally the
          investments. Therefore another approach is used when high
          resolution is called for.\\

                                    {\bf Table 4.2}\\


             \begin{tabular} {||l|l|l||} \hline
          X-ray Line & Shift (eV) & Relative Intensity (\%)\\ \hline
          Mg$_{K_{\alpha 12}}$ & 0.0 & 100 \\ \hline Mg$_{K_{\alpha
          '}}$ & 4.5 & 1.0 \\ \hline Mg$_{K_{\alpha 3}}$ & 8.4 & 9.2 \\
          \hline Mg$_{K_{\alpha 4}}$ & 10.0 & 5.1 \\ \hline
          Mg$_{K_{\alpha 5}}$ & 17.3 & 0.8 \\ \hline Mg$_{K_{\alpha
          6}}$ & 20.5 & 0.05 \\ \hline Mg$_{K_{\beta}}$ & 48.0 & 2.0
          \\ \hline
          Al$_{K_{\alpha 12}}$ & 0.0 & 100  \\ \hline Al$_{K_{\alpha '}}$ &
          5.6 & 1.0  \\ \hline Al$_{K_{\alpha 3}}$ & 9.6 & 7.8  \\
          \hline Al$_{K_{\alpha 4}}$ & 11.5 & 3.3  \\ \hline
          Al$_{K_{\alpha 5}}$ & 19.8 & 0.4  \\ \hline Al$_{K_{\alpha
          6}}$ & 23.4 & 0.3  \\ \hline Al$_{K_{\beta}}$ & 70.0 & 2.0  \\
          \hline \end{tabular}

           \vspace{1cm}


             By using synchrotron facilities it is possible to obtain an
          extremely brilliant source of X-rays in a broad spectrum of
          energies (10-1000 eV). The radiation is formed by confining
          high energy  (5  GeV)  electrons  into  a  storage  ring,  a
          so-called synchrotron. Electrons, which are accelerated, will
          emit radiation. By establishing suitable  monochromators  it
          is possible to select the desired photon  energy.  This  has
          experimentally many advantages since the photon  energy  is
          tuneable (so it is possible to select a photon energy  where
          maximum  surface  sensitivity  can  be  obtained)  and   the
          broadening can be controlled. The  only  major  drawback  on
          this type of equipment is that  it  is  huge  and  expensive
          facilities, located only a few places in Europe.


\section{Spectral Interpretation}

          Now all the equipment has been established for doing the XPS
          experiment. A typical XPS spectrum of Cu(100) is shown in
          Figure 4.7 where the source is the Al$_{K_{\alpha 4}}$
          radiation from a conventional dual Al/Mg X-ray source. The
          observed spectrum is a mapping of the energy levels in
          copper. The features at the highest kinetic energy i.e. zero
          binding energy are from the valence region of copper. This
          is not very intensive and can hardly be seen.\\

\vspace*{12cm}

\noindent          {\bf Figure 4.7} XPS spectrum of a Cu(100) surface measured
          by use of a HSA analyser.\\

             What is seen is the 3d level lying just below the Fermi
          level. Then follows the 3p and 3s levels around 65 and 110
          eV binding  energy  respectively.  The  Auger  lines  follow
          naturally from the relaxation of the  holes  that  has  been
          created by the photoemission  process.  They  are  not  very
          interesting  in  this  context  and  we  shall   leave   the
          discussion of these to the  next  chapter.  At  even  higher
          binding energies we find first  the  2p$_{\frac{3}{2}}$  and
          then the 2p$_{\frac{1}{2}}$ lines. Contrary to the  3p  line
          the spin orbit splitting is  very  obvious  and  amounts  to
          roughly 20 eV.  This  splitting  will  be  present  for  all
          subshells with an angular momentum higher  than  zero.  Thus
          the only lines where no splitting should be expected are the
          lines due to excitations of s electrons. For careful studies
          of the surfaces it may be informative to take a closer  look
          at the high energy region of the spectrum as shown in Figure
          4.8. It is now easily seen that this surface is contaminated
          as all the lines cannot be accounted for by  copper.  Carbon,
          oxygen and sulphur are  easily  identified  in  non-vanishing
          amounts.  Furthermore,  it  is  also  possible  to  identify
          ghost-peaks from the Mg$_{K\alpha}$ anode, which are excited
          when the aluminium source is used. Even some $\beta$
          satellites from the Cu$_{2p}$ lines can be observed. All
          such features has to be accounted for in a detailed XPS
          study.\\

\vspace*{12cm}

          \noindent   {\bf Figure 4.8} As Figure 4.7 but rescaled and only
          showing the region above 965 eV kinetic energy.\\





             \subsection{Multiplet splitting in XPS}
          To understand the various features observed by XPS we
          must look a little into atomic physics. In the following we
          shall treat all atoms as if they initially have filled
          shells, i.e. we will disregard the valence electrons which
          are highly delocalised in metals. All the electron energy levels
          can now be determined by finding at solution to the
          Schr�dinger equation with the appropriate Hamiltonian. This
          can be done iteratively by a Self Consistent Hartree Fock
          calculation. In this case all electron energy levels like 1s,
          2s, 2p, 3p, and 3d for the copper atom will be degenerate.
          This state will in the following be referred to as the
          {\em initial state}. This degeneracy of a level will be
          lifted if an electron is being excited out of the atom
          leaving a hole in a subshell say 2p. The same calculation
          has to be performed for this system, now referred to as the
          {\em final state}, but now also the spin-orbit interaction
          has to be taken into account since we have an unpaired hole
          \footnote{A missing electron in an otherwise full shell is
          equivalent to having only one electron in an otherwise empty
          shell, except that the spin-orbit interaction changes sign.}.
          The spin-orbit coupling is treated as a perturbation of the
          system and in the present case it is convenient to formulate
          a new quantum number j instead of the angular momentum l and
          the spin s as $\vec{j}$=$\vec{l}$+$\vec{s}$. As $\vec{s}$
          only can take the values +$\frac{1}{2}$ and -$\frac{1}{2}$
          and as j only can take the values \begin{equation} l+s \geq
          j \geq \mid l-s \mid \end{equation} it is obvious why there
          will always  be a splitting for levels with l$\neq$0. The
          intensity distribution in the two lines will be given by the
          degeneracy in the two final states which is 2j+1.  Thus  the
          intensity ratio between the two 2p lines observed in  Figure
          4.7 is 4:2. The same formalism can be used to understand the
          splitting observed in the 2p, 3d, and  4f  shells  shown  in
          Figure 4.9 \cite{siegbahn1}.\\

\vspace{1cm}

\noindent            {\bf Figure 4.9} The XPS spectra\\ of a variety of elements
        showing\\ the development in spin-orbit\\ splitting for the 2p,
       3d, and\\ 4f levels.\\

\vspace*{8cm}



             This was for the simple case where open shells could be
          neglected, but that is not always the case. If we consider
          the rare earth metals (which by no means are rare) the 4f
          shell will be very localised on the atom although it is
          weakly bound. This is due to the lanthanide contraction and
          the same phenomena applies to the actinides. Thus when going
          through the lanthanide's series where the 4f shell is being
          filled, the chemistry is not changing because the outermost
          electron configuration is basically not changing. This also
          has a strong influence on the observed XPS spectra of  these
          metals.  Figure  4.10  shows  one  of  the  relative  simple
          examples of multiplet splitting \cite{chorkendorff}.\\

\vspace{13cm}



            \noindent {\bf Figure 4.10} XPS spectrum of a) pure
        Ytterbium and b) an Ytterbium-Nickel alloy. The full drawn
       line represents the background due to inelastic energy
     losses during the transport.\\


             In the top panel of Figure 4.10  the 4d region of pure
          Ytterbium is shown. As it  has  the  electron  configuration
          [Xe]4f$^{14}$6s$^{2}$ it is a divalent metal with  otherwise
          filled shells. Thus the 4d region should just show a  simple
          4d$_{\frac{5}{2}}$  and  4d$_{\frac{3}{2}}$  splitting.   At
          first glance it looks like it  does  not  have  the  correct
          intensity distribution, but this can easily be explained  by
          the background of electrons that has undergone energy losses
          from deeper layers. The background can be estimated as shown
          as a solid line in Figure 4.10 by studying the  energy  loss
          spectra of pure Yb. In order to get  a  clear  view  of  the
          feature it has been subtracted and the ``true'' spectrum  is
          shown in Figure 4.11 together with a fit taking into account
          the  appropriate  X-ray  satellites   \cite{chorkendorff}.\\
          \vspace{13cm}

            \noindent {\bf Figure 4.11} The background corrected
        spectra of Figure 4.10 shown together with theoretical fits
        relying on multiplet splitting.\\

             The fit is quite reasonable although not perfect. The
          reason for discrepancy is that intrinsic energy loss
          mechanism are not included in the background subtraction
          which is only dealing with the transport of electrons out of
          the material.




             Now if Ytterbium is evaporated onto Nickel and the sample
          is heated slightly an alloy is formed and the Yb 4d region
          changes completely as shown in the lower panel of Figure 4.10
          and Figure 4.11. This change can be understood in terms of a
          change of valence of Yb from being divalent to become
          trivalent. A trivalent Yb atom will initially have a hole in
          the 4f shell so the final state after the photoemission
          process will be an atom with a hole both in the 4f shell and
          one in the 4d shell. These two holes will interact strongly
          through electrostatic interaction. This sort of interaction
          is treated well though the so-called Russell Saunders
          coupling scheme or LS-couplings scheme where the two angular
          moments l$_{1}$ and l$_{2}$ are coupled to L and similar for
          the two spin values. This immediately leads to large numbers
          of different final states, L=1,2,3,4,5 and S=0,1, which
          usually are described though the terms\\

 \vspace{0.3cm}

     \noindent     $^{1}P, ^{3}P\\ ^{1}D, ^{3}D\\ ^{1}F, ^{3}F\\ ^{1}G, ^{3}G\\
          ^{1}H, ^{3}H $\\

          \vspace{0.3cm}

           a number of singlet or triplet
          states ($^{(2S+1)}X$). All these ten terms have different
          energies and will therefore appear in the spectrum. However,
          this is not all. We have not considered the spin orbit
          coupling. When that is taken into account L and S are no
         longer good quantum numbers. By treating the spin-orbit
          coupling as a perturbation a new set of eigenvalues can be
          developed on the basis set of L and S states and the new
          coupling scheme will be the so- called intermediate coupling
          scheme where $\vec{J} = \vec{L}+ \vec{S}$ are the good
          quantum numbers. The result of this extra perturbation is
          that the degeneracy of the triplet states is lifted and each
          of these splits up in three different states. There will now
          be 20 final states\\

\vspace{0.3cm}


    \noindent      $^{1}P_{1}, ^{3}P_{0}, ^{3}P_{1}, ^{3}P_{2}\\ ^{1}D_{2},
          ^{3}D_{1}, ^{3}D_{2}, ^{3}D_{3}\\ ^{1}F_{3}, ^{3}F_{2},
          ^{3}F_{3}, ^{3}F_{4}\\ ^{1}G_{4}, ^{3}G_{3}, ^{3}G_{4},
          ^{3}G_{5}\\ ^{1}H_{5}, ^{3}H_{4}, ^{3}H_{5}, ^{3}H_{6} $\\

          \vspace{0.3cm}

             The intensities for the various final states are not given
          by a simple expression due to the mixing of various LS
          states. All these and the appropriate $\alpha_{34}$ X-ray
          satellites are taken into account in Figure 4.11 and a good
          fit is obtained. Thus XPS is very useful for determination
          of the electronic configuration in such cases.

             Figure 4.12 shows the valence band region of Yb studied
          with a synchrotron facility where the photon frequency is
          tuneable \cite{gerken}. The valence  electrons  (6s6p)$^{2}$
          can just be observe for the lowest photon energies as a step
          function at the Fermi level. In this region we should expect
          a simple doublet resulting from the filled 4f shell, however
          two doublets are observed.\\

             \vspace*{12.5cm}
          \noindent {\bf Figure 4.12} The valence band region of
          Ytterbium excited by synchrotron light with increasing energy \cite{gerken}.\\


             By increasing the photon energy it is clearly seen that
          the doublet at the higher binding energy disappears. If we
          choose to look at the same region with XPS we actually only
          observe a single doublet. This shows how the surface
          sensitivity changes with energy see Figure 4.13 \cite{gerken}. The
          electrons are in this case essentially emitted with the
          photon energy and we should therefore expect low surface
          sensitivity when the photon energy is high. Thus the doublet
          observed at high energy is due to signal from the bulk of
          the material, whereas the extra doublet observed at low
          photon energy is due to the surface region. As we shall see,
          the bonding energy of the electron is dependent not only  on
          the atom and its electron configuration,  but  also  on  the
          chemical surroundings of the atom. In this case the  surface
          atoms are missing a number of neighbours compared  to  those
          sitting in the bulk. They will  therefore  be  located  in  a
          different electron density which will change the  energy  of
          the emitted electron  from  the  bulk  value.  The  observed
          spectra shown in Figure 4.12 allows for an estimation of  the
          inelastic mean free path which in the present case was found
          to be roughly 4 � at 20 eV and 10 � at 180 eV.

\vspace*{12cm}

             \noindent {\bf Figure 4.13} Comparison between the
          valence band region of Ytterbium excited by 100 eV photons
          and by al$_{K\alpha}$ radiation \cite{gerken}.\\

             The alloying and change in electron configuration of the
          Yb could just as well have been studied by following the
          region of the 4f electrons. This demonstrates how surface
          reactions between metals can be followed quite nicely
          especially if monochromatised synchrotron light is used.




             \subsection{The Binding Energy in XPS}




          Since the binding energy of the various shells is
          characteristic for each element, as can be seen from Figure
          4.14, this in itself allows for identification of the elements. In
          the previous section we assumed that the kinetic energy of
          the emitted electron was given by \begin{equation}
          E_{kin}=h\nu-E_{B}-\Phi \end{equation}

\vspace*{13cm}
          \noindent {\bf Figure 4.14} The binding energy for a broad
          variety of levels as a function of atomic number \cite{siegbahn1}.\\

                                                 In the following we
          shall disregard $\Phi$ and concentrate on the binding energy
          E$_{B}$.  It  was  earlier   assumed   that   the   electron
          configuration of the atom (except for  some  splitting)  did
          not change by the photoemission i.e. the energy of all other
          electrons is the same as before the  photoemission  process.
          This energy is referred to as the Koopman's energy. However,
          this is never observed  because  the  atom  is  a  dynamical
          system which  immediately  will  relax  during  the  process
          screening out the hole made in the core level.  Thus  higher
          lying electrons will in principle feel the core change  from
          Z to approximately Z+1 whereby they  will  obtain  a  higher
          binding  energy.  The   relaxation   will   result   in   an
          photoelectron with a higher  kinetic  energy.  The  gain  in
          energy  we  shall  call  E$_{ar}$,  the  atomic  relaxation.
          Including this term the kinetic energy will for an atom be

\begin{equation}
E_{kin}=h\nu-E_{B}+E_{ar}
\end{equation}


          This is only true if the photoemission process is an
          adiabatic process, i.e. the emission process is slow enough
          that equilibrium is obtained. This is not true for the
          photoemission process where the sudden approximation scheme is
          much more applicable. All electrons will feel a sudden
          change in potential which may sometimes lead to excitations
          of other electrons in the atom. As the ionised atom may now
          be left in an excited state there will be less energy for
          the emitted photoelectron and a lower kinetic energy will be
          measured. The result will therefore be that a number of
          satellites which will be observed towards lower
          kinetic energy. An example of such a phenomena is shown in
          Figure 4.15 where the Xe 3d spectrum is shown.\\

\vspace{13cm}

             \noindent {\bf Figure 4.15} The 3d region of atomic Xe
          clearly showing shake up-satellites \cite{siegbahn1}.\\


          The two parent lines are easily identified, but notice the
          rich structure towards the higher binding energy, which
          is mainly due to  a  simultaneous  excitation  of  the  5p
          electrons to various empty levels. This process is called  a
          shake-up process as  the  electrons  are  excited  to  bound
          states. If they are excited to the continuum they are called
          shake-off satellites. The real origin of this  phenomena  we
          shall find in the many-body description of the atom.  In  an
          accurate description of an atom it is  usually  necessary  to
          invoke Configuration Interaction (CI). Neither  the  initial
          state ($\Psi_{i}$) nor the final  state  $\Psi_{f}$  can  be
          described  by  a  single  electron  configuration  but  are
          described by a  linear  combination  of  wave-functions  for
          different configurations with different energies. Thus  when
          forming  the  matrix  element  for  finding  the  transition
          probability there will be possibilities for  some  atoms  to
          have strong components of various electron configurations as
          seen in  Figure  4.15.  In  most  cases  only  configuration
          interaction has to be considered for the  final  state.  For
          further details on this effect  the  reader  is  referred  to
          \cite{shirley}.

             Let us now consider an atom embedded in a metal.
          Naturally we will still have the atomic relaxation but now
          also another effect will influence the kinetic energy of
          the photoelectron. The weakly bound and delocalised valence
          electrons will be able to respond swiftly to the potential
          changes set up by the photoemission process. Charge will
          flow in and screen the emitted electron from the ionised
          atom resulting in a higher kinetic energy,, see Figure 4.16.\\

\vspace{12cm}

             \noindent {\bf Figure 4.16} Schematics showing the effect
          of extra atomic relaxation in a metal.
          \vspace{1cm}

             This contribution we shall call extra atomic relaxation
          E$_{ear}$ whereby the kinetic energy is given by

\begin{equation}
E_{kin}=h\nu-E_{B}+E_{ar} +E_{ear}
\end{equation}
             The size of E$_{ar}$ and E$_{ear}$ can be up to 10-15 eV
          so a substantial difference between the spectra of an
          element in the vapour phase and in the solid state can be
          observed see Figure 4.16.
          Just as in the atomic case the sudden approximation may lead
          to excitation of various states. As we are considering
          a solid there will be no discrete levels above the Fermi level
          but a band structure so that the CI-picture is not
          appropriate. Instead we can just consider some of the energy
          loss mechanisms we discussed in chapter 2. Excitations of
          excitons is equivalent to the shake up- satellites and also
          excitations of plasmon may be observed. These intrinsic
          energy losses have to be distinguished from those observed in
          conjunction with the transport of the electrons through the
          solid which we shall call extrinsic energy losses. In Figure
          4.10 we subtracted a background due to the transport of the
          electrons through  the  Yb.  However,  some  discrepancy  is
          still observed in  the  background  corrected  spectra  when
          fitted  with  a  theoretical  line   shape.   The   observed
          discrepancy can very  well  be  explained  by  a  so-called
          intrinsic energy loss to a plasmon  created  by  the  sudden
          potential change during  the  photo  emission  process.  The
          effect of intrinsic excitons are naturally very difficult to
          separate experimentally from the  extrinsic,  but  they  can
          easily be observed in for instance the Pt 4f spectrum  shown
          in Figure 4.17 \cite{wertheim}.\\

\vspace{9cm}

             \noindent {\bf Figure 4.17} The 4f region of platinum
          displaying an asymmetrical peak shape.\\

             A priori a Lorenzian line shape is to be expected as the
          broadening primarily is determined by the lifetime of the
          final state. It is, however, clearly seen that the two lines
          are slightly asymmetrical towards lower kinetic energy.
          Usually this sort of lines can be fitted by a Doniach-Sunjic
          line shape which is asymmetrical \cite{doniach} and
          explained by excitations of excitons.

             The separation of the lines in their different
          components is very important for obtaining a high accuracy
          on quantitative determinations of surface compositions. It is
          therefore very important to understand the origin of the
          various effects.

             So far  we have only discussed the influence of the
          final state on the binding energy. Different chemical
          surroundings will also affect the initial state as the
          electron configuration and density around the atom will be
          changed. This change in binding energy related to various
          chemical surroundings is called the chemical shift. However,
          the final state will also be sensitive to the chemical
          state of the atom. Thus the phenomena can only be separated
          in an initial and a final state effect from a theoretical
          point of view. The kinetic energy can formally be written as

\begin{equation}
E_{kin}=h\nu-E_{B}+E_{ar} +E_{ear}+E_{chem}
\end{equation}

             where E$_{chem}$ is dependent on the
          surroundings of the atom.

          An example of a chemical shift is illustrated in Figure 4.18
          where the 2p region of Ti and TiO$_{2}$ are depicted \cite{perkin}.


             It is clearly seen that the Ti 2p lines are shifts
          roughly 4.5 eV towards higher binding energy by oxidation
          into TiO$_{2}$. This sort of data has been measured for a
          number of chemical compounds and the results are shown in
          Figure 4.19 for Ti \cite{perkin}.


             Notice how the binding energy increases with increasing
          electronegative neighbours. Such information  only gives
          an indication of the chemical state of a surface, but in
          combination with the possibility to identify which elements
          are present else and how much, this can be  valuable
          information. One area where the chemical shift plays a
          strong role is with in the field of polymer research where
          it can be used to identify the role of functional groups and
          their behaviour by adhesion. As an example  the XPS
          spectrum of ethyltrifluroacetate is shown in Figure 4.20 where
          4 lines of carbon are easily  identified \cite{ghosh}.

             Sometimes the chemical shift will also be followed by a
          change in the availability to excite different satellites. This
          is clearly seen from Figure 4.21 where the copper 2p region
          undergoes a chemical shift of roughly 1 eV by oxidation, but
          at the same time a strong feature with equal intensity to
          the main line is observed at 10 eV higher binding energy \cite{perkin}.
        \newpage
\vspace*{19cm}
\noindent {\bf Figure 4.18} The chemical shift of Ti upon oxidation \cite{perkin}.

                \newpage
\noindent {\bf Figure 4.19} The Ti 2p\\ line position
         for\\ various chemical\\ compounds \cite{perkin}.\\
         \vspace*{18cm}


         \noindent {\bf Figure 4.20} The chemical shifts of
          carbon in ethyltrifluroacetate.

                                         \newpage


            \noindent {\bf Figure 4.21} The chemical\\ shift of
        copper and\\ the satellite\\ structure \cite{perkin}.\\

\vspace{12cm}



              \subsection{Line Width}



          Especially when chemical shifts and  electron  structures  are
          studied it is important to consider the line width of the XPS
          lines. Ideally  both the energy  distribution  of  the
          X-ray  line  and  the  energy  distribution  of  an  emitted
          electron  should be given by the lifetime of the holes left behind (if
          we disregard any contribution from intrinsic  and  extrinsic
          energy  losses)  and  the  line shape  should  therefore   be
          described by a Lorenzian which has the form
          \begin{equation}
          L(E)= \frac{A}{(E-E_{0})^{2}+\frac{\Gamma^{2}}{4}}
          \end{equation}
          where $\Gamma$ corresponds to FWHM. If there were  no  other
          sources of  broadening    the  lines  than  these  two,  the
          resulting XPS line shape would be given by

          \begin{equation}
          P(E)= L_{X-ray}(E)*L_{Photoemission}(E)
          \end{equation}
          where * refers to a convolution of the two  line shapes.  The
          analyser  will,  however,  also  always  contribute  to  the
          broadening of the observed lines. This broadening can usually
          be described adequately by a Gaussian line shape
                    \begin{equation}
          G(E)=           Be^{-\frac{(E-E_{0})^{2}4ln(2)}{\Gamma^{2}}}
          \end{equation}
          where $\Gamma$ again corresponds to FWHM.  Thus  in  order  to
          find the resulting line shape the line shape P(E) also has  to
          be convoluted by G(E)
             \begin{equation}
                          R(E) = P(E)*G(E)
                          \end{equation}

          The resulting  line shape  R(E)  can  be  approximated  by  a
          Gaussian and the resulting FWHM can be approximated by
                    \begin{equation}
\Gamma_{res}=\sqrt{\Gamma_{X-ray}^{2}+\Gamma_{Photoemission}^{2}+\Gamma_{Analyser}^{2}}
          \end{equation}
          If    for    example    $\Gamma_{X-ray}$=0.85     eV     and
          $\Gamma_{Photoemission}^{2}$=0.5 eV it  would  then  not  be
          meaningful to press the  resolution  of  the  analyser  much
          below 0.5 eV since this would just lead to  loss  of  signal
          without any significant gain in resolution.
          As the signal to noise ratio ideally will be proportional  to
          $\sqrt{Intensity}$ it is important always to counter-balance
          resolution and intensity.


          \subsection{The Transition Probability}

          The probability for excitation of an electron from  an  atom
          when interacting with a X-ray photon  can  be  described  in
          quantum mechanics as  an  interaction  between  the  initial
          and  final  state  through  an  interaction  Hamiltonian.  The
          interaction  Hamiltonian  can   be   approximated   by   the
          time-dependent electrical field and the problem can be solved
          by use  of  time-dependent  perturbation  theory  leading  to
          Fermi's Golden rule:
                            \begin{equation}
          P  \propto  |<\Phi_{initial  state  }|H_{int}(t)|\Phi_{final
          state}>|^{2}
          \end{equation}

          By  such  calculation  it  is  possible  to   estimate   the
          cross-section per X-ray photon  to  excite  a  photoelectron.
          The cross section is dependent  on  the  photon  energy  and
          is, as can be seen from Figure 4.22  also very dependent on which
          shell is considered. The above calculation relies on
          the assumption that the initial and final states can be
          described by one electron wave functions, though it should not be
          expected to be very accurate. It will for example not be able
          to describe many-body effects like the shake-up and shake-off
          satellites, but it gives a good indication of which
          photoemission lines are accessible.\\
             \vspace{1cm}

             \noindent {\bf Figure 4.22} The calculated\\ cross-sections
          for 1.5 kV\\ photons for various\\ levels as a function\\ of
          atomic number \cite{shofield}.\\

          \vspace{12cm}


              It is obvious, from Figure 4.21, that it
          will be very difficult (= impossible) to measure the
          photoemission lines of hydrogen and helium by Al$_{K\alpha}$
          radiation. Concerning the satellites, there exits a sum rule
          saying that the cross section determined by the above one
          electron approximations equals the cross section for all the
          possible satellites and the parent line in the many body
          picture. Thus if strong satellites are involved like seen in
          the copper oxide, special care should be taken using these
          calculated cross sections. Due to the various uncertainties of
          the calculated cross section, it is much more common but not
          without problems to rely on empirical parameters for
          quantitative analysis by XPS.








          \section{Quantitative XPS Analysis}

          When a surface  is  exposed  to  X-rays  there  will  be  the
          following probability for emission of a photoelectron  per
          incident photon
          \begin{equation}
          P_{xk} = \sigma_{xk}N_{x}t
          \end{equation}
          where $\sigma_{xk}$ is the cross section of level k  in  the
          element x, N$_{x}$ is the concentration (atoms/cm$^{3}$)  of
          element x in the surface, and t  is  the  thickness  of  the
          probed area. The probability for exciting electrons in the
          sample is not the interesting number, what is interesting
          is the probability for those electrons to escape the surface
          without undergoing energy losses. If we assume that the
          surface is homogeneous (N$_{x}(z)$ = N$_{x}$ for t >>
          $\lambda_{xk}$) the probability for getting the electrons
          outside the sample, emitted from a depth of z, can be
          approximated by

         \begin{equation}
          dP_{xk}'(z) = \sigma_{xk}N_{x}e^{-\frac{z}{\lambda_{xk}}}dz
          \end{equation}

             which by integration to all depths leads  to

         \begin{eqnarray}
          P_{xk}'   &   =   &   \sigma_{xk}N_{x}    \int_{0}^{\infty}
          e^{-\frac{z}{\lambda_{xk}}}dz\\
                   P_{xk}'     &     = &
          \sigma_{xk}N_{x}\lambda_{xk}
           \end{eqnarray}

             Here $\lambda_{xk}$ is the inelastic mean free path for
          an electron emitted from level k travelling through material
          x. Thus the probability of emitting electrons beyond an
          infinite thick sample will always be the same as if we were
          looking at a sample of thickness $\lambda_{xk}$ without any
          damping. The X-ray will always penetrate deeply into the
          surface and excite atoms, but the rather small mean free
          path for the photoelectron ensures the surface sensitivity
          as indicated in Figure 4.23


          The number of electrons detected from  shell  k  of  element
          x will, besides the above probability, be proportional to  the
          number of photons F$_{h\nu}$ and the  transmission  function
          for the analyser used T(E$_{kx}$)

            \begin{equation}
          I_{xk} = \sigma_{xk}N_{x}\lambda_{xk}F_{h\nu}T(E_{xk})
          \end{equation}

             \noindent This is the important formula for determining
          the composition of surface which is homogeneous.\\

          The intensity contribution from a layer in the depth $z$ of
          thickness $dz$ will be

\begin{eqnarray}
dI_{xk} & = & dP_{xk}'(z)F_{h\nu}T(E_{xk})\\
        & = & \sigma_{xk}N_{x}e^{-\frac{z}{\lambda_{xk}}}F_{h\nu}T(E_{xk})dz\\
        & = & \frac{I_{xk}}{\lambda_{xk}}e^{-\frac{z}{\lambda_{xk}}}dz
\end{eqnarray}



       \subsection{Example}

         Estimate the surface composition of an alloy
          made of NiSi assuming it to be homogeneous. The Intensity
          ratio $\frac{I_{Si2p}}{I_{Ni3p}}$ is found to be 0.20. The
          cross section for the two lines can be found in Figure 4.22
          to be 1*10$^{-20}$ and 2.5*10$^{-20}$ cm$^{-2}$ for Si and Ni
          respectively. Since the binding energy for the two lines is
          nearly the same, see Figure 4.14 it is safe to assume that
          \begin{equation} \frac{I_{Si2p}}{I_{Ni3p}} =
          \frac{\sigma_{Si2p}N_{Si}\lambda_{Si2p}F_{h\nu}T(E_{Si2p})}{\sigma_{Ni3p}N_{Ni}\lambda_{Ni3p}F_{h\nu}T(E_{Ni3p})}
          \end{equation} can be approximated by

\begin{equation}
          \frac{I_{Si2p}}{I_{Ni3p}}                                  =
          \frac{\sigma_{Si2p}N_{Si}}{\sigma_{Ni3p}N_{Ni}}
          \end{equation}

             whereby the atomic composition can be estimated
          suggesting a Ni$_{2}$Si alloy.\\

\vspace{1cm}

            \noindent {\bf Figure 4.23} Schematics\\ showing the
        difference\\ in penetration depth\\ of electrons and X-rays.\\

\vspace{9cm}


          The mean free  path  can  alternatively  be  estimated  from
          figure 2.6 and the transmission function can be determined
          so this method can be  used  more  generally.  However  these
          different parameters are often put together in what is
          called a sensitivity factor S$_{xk}$ which then are
          determined for a certain type of analysers. The intensity of
          a line can then be determined as the area of the XPS line
          (or less accurate simply the peak height) and would then be
          given by


            \begin{equation}
          I_{xk} = S_{xk}N_{x}F_{h\nu}
          \end{equation}
          and the atomic concentration of each element present in the
          sample can then be estimated as

            \begin{equation}
          C_{x} =
          (\frac{I_{xk}}{S_{x}}*100\%)/(\sum_{i=1}^{N}\frac{I_{i}}{S_{i}})
          \end{equation} where i refers to only one subshell in any of the N observed
          elements. Also in this case it is assumed that the surface
          is homogeneous. More elaborate estimations can be done
          without this assumption, but they will always be model
          dependent.







          The surface sensitivity can be improved considerably
          if the surface of the  sample  is  tilted  relative  to  the
          analyser as  shown  in  figure  4.24.\\
          \vspace*{8cm}


            \noindent {\bf Figure 4.24} Schematics showing the
        effect of a  take-off angle $\theta$.\\



    In this manner the electrons emitted from
          the depth z will now have to pass through a layer of
          thickness $\frac{z}{cos(\theta)}$ where $\theta$ is the
          angle between the surface normal and the analyser. If
          we consider a thin over layer of material y of thickness
          d on top of layer x equation we will by integration of eqn.
          4.21 get 

            \begin{equation}
           I_{xk}(d)  =  I_{xk}e^{-\frac{d}{cos(\theta)\lambda_{yk}}}
          \end{equation}

          and \begin{equation}
           I_{yk'}(d)  =  I_{yk'}(1-e^{-\frac{d}{cos(\theta)\lambda_{yk'}}})
          \end{equation}

Thus by varying the angle it is possible to determine if
          the surface is inhomogeneous in the outermost layers or if
          it has another electron structure as we saw it in the
          Ytterbium case. Naturally surfaces are not homogeneous and
          especially at the surfaces there are several reason why this
          is not the case. For example the spectra of the Cu(100)
          single crystal shown in Figure 4.7 and 4.8 clearly show
          traces of contamination's that will be confined to the
          surface. The oxygen and carbon are due to contamination
          since the crystal has been standing in the vacuum chamber
          for an extended period without cleaning. The presence of the
          sulphur, however, is due to segregation from the bulk of the
          crystal, which under the heat treatment necessary to anneal
          the crystal and re-establish the order after the cleaning
          procedure (argon sputtering) has diffused out at the
          surface. The amount of sulphur in the crystal amounts to the
          ppm level, but as the sulphur has a much lower surface
          energy than copper it is energetically more favourable for
          the system to have the sulphur on the surface. This is a
          quite general phenomenon and most alloys will therefore be
          enriched in one component at the surface compared to the
          bulk values. 







             The above formalism can be refined in  several  ways  see
          \cite{briggs} and references  therein.  We  shall  not  deal
          further with these problems here, but just mention  that  one
          of the major problems when performing quantitative analysis by
          XPS is the subtraction of the background in the spectra. The
          intensity mentioned above is proportional to the area of the
          XPS line, thus the underlying background  due  to  electrons
          that have undergone energy losses should be eliminated  just
          as it was done in Figure 4.7. In general  the energy  loss
          function is not known well enough that such a procedure can  be
          used, and even when it was  we  saw  that  there  was  still  a
          discrepancy because of the  intrinsic  energy  losses.  Thus
          quantitative analysis with XPS will always have a rather large
          uncertainty and results better than  10\%  should  never  be
          expected. The satellites will always be a source  of  errors
          and the assumption of a homogeneous sample is  usually  only
          fulfilled for pure elements. Nevertheless,  XPS  has  proven
          extremely useful in many cases and there exists today no
          other method for determining  the  surface  composition  as
          accurate as XPS.

                   \newpage
\section{Problems}



\begin{enumerate}

             \item What is the basic reason for the surface sensitivity
          of the electron spectroscopic methods?

             \item Are there elements which can not be detected by
          XPS? AES?



             \item We wants to determine the mean free path of
          electrons in iron by using the method described in the
          notes. Thus iron is deposited on a substrate of Ni and the damping
          of the Ni lines are measured. Two photoemission lines
          from Ni are used (excitation by $Al_{k\alpha}$) which are
          observed at 65 eV and 700 eV binding energy, respectively.
          Due to the geometry of the experimental set-up will only
          electrons be detected in a little solid angle positioned
          $60^{\circ}$ from the surface normal. The intensity
          of the XPS signal before any deposition was measured
          to be $I_{65 eV} = 10000 c/s$ and $I_{700 eV} = 60000
          c/s$. Now $ 8.6*10^{-7}$ g iron pr. $cm^{2}$ is deposited
          whereby the to lines are reduced to $I_{65 eV} = 2300 c/s$
          and $I_{700 eV} = 6600 c/s$. Under which assumption can we
          estimate the mean free path of the electrons in iron?
          Make the assumption and estimate the mean free paths.

             Suggest/discuss a method to estimate when $ 8.6*10^{-7}$ g of
          iron pr. $cm^{2}$ has been deposited.



             \item XPS is used to check whether a Ni(100) single
          crystal is clean enough, that it can be used for surface
          reactivity experiments. The analyser measure along the
          surface normal. The acquired spectra reveals a small
          carbon feature ($C_{1s}$) which is measured relatively to
          the $Ni_{3p}$ line. The ratio is determined to be  0.018.
          We shall now interpretate these data by use of two models.
          In the first model we shall assume that all the carbon is
          located on top of the Ni surface with coverage $\Theta$ less
          than one relative to  the Ni atoms. The mean free path for the two lines will be assumed to be equal namely
 $\lambda = 15 �$. Is that a serious approximation ? The cross sections
          for the two lines $C_{1s}$ and $Ni_{3p}$ can be read from
          Fig. 4.22 Determine within this model the coverage $\Theta$
          of carbon. Will a further cleaning of the surface be
          necessary or can we continue our investigations on the
          behaviour of clean Ni surfaces?

             In the second model we assume that the carbon is
          distributed homogeneously into the Ni. Determine the carbon
          concentration and especially the carbon concentration at the
          surface.

             Suggest an experiment that will allow us to determine
          which of the two models that is correct.



             \item Explain the observed intensity distribution in the
          enclosed figure 4.25 of clean Samarium excited with
          different photon energies. Hint: The photoemission lines
          below 3 eV binding energy are due to divalent Samarium,
          whereas the lines above 3 eV binding energy are due to
          trivalent Samarium. Samarium is considered to be trivalent
          in the metallic state.





             \item Figure 4.26 displays the uptake curves of carbon on a
          Ni single crystal when it is exposed to CH$_{4}$ at
          different temperatures \cite{chorkendorff2}. The saturation coverage
          is known to be 0.5 carbon atom per Ni atom. The carbon
          coverage as a function of dosage was determined with
          XPS. Determine from this set of data the sticking
          probability of methane on this surface in the zero coverage
          regime at various temperatures and estimate the activation
          energy for this process. Why are the curves not straight lines?
          If it took one hour to give the longest exposure, what is
          then the methane pressure over the crystal?





\end{enumerate}


\vspace{12cm}
          \noindent {\bf Figure 4.25} Synchrotron and XPS  spectra  of
          pure samarium \cite{gerken}.

          \newpage
          \vspace*{13cm}
          \noindent {\bf Figure 4.26} Carbon coverage on Ni(100) as  a
          function of methane dosage at various temperatures \cite{chorkendorff2}.

