\newpage
\chapter{Adsorption Processes}
\section{Introduction}

The gas-surface interaction and reactions on surfaces plays a very important role in many technologically important areas such as in corrosion, adhesion, synthesis of new materials, heterogeneous catalysis, etc. and  is therefore desirable to obtain a profound insigth in these processes. The impact of these studies have maybe been strongest in the field of heterogeneous catalysis where it has been possible to study quite realistic model systems since a large number of catalysts actually can be modelled by studies on well defined single crystals. Despite many years of effort do there, however, only exists very few examples, see fx \cite{Sciencepaper}, that surface science have let directly to the invention of a new catalyst. On the other hand has surface science provided the framework and necessary tools for understanding and designing new heterogeneous catalysts. The essential step in this process is first adsorption of the reactant on to the surface. Here they may react much easier with other reactants due to the changed chemical environnement on the surface. The proces should then be followed by a desorption process of the products cleaning of the surface making it available for new reactancts. In this part we shall focus on the physics and chemistry involved when gasses adsorbs on the surface and in particular metal surfaces. 

When an atom or molecule approaches the surface it will fell the potential energy surface set up by the metal atoms in the solid. The interaction is usually divided into two regime the physisorption regime and the chemisorption regime.

\section{Physisorption}

Physisorption is a weak interaction which is characterized by lack of a true chemical bond between the adsorbate and the surface, i.e. no electrons are shared. The physisorption interaction  is conveniently divided into to parts: A strongly repulsive part at close distances and a Van der Waals interactions at medium ranges.
\subsection{The van der Waals interaction}
When a  atom/molecule approaches a surface will the  electrons in the atom/molecule, due to quantum fluctuations,  set up  an dipole which induces an image dipole in the polarizable solid. Since this image dipole is of the oppersite sign and correlated with the fluctuation in the atom/molecule will the resulting force be attractive. In the following we shall construct a simple model which elucidates this phenomena. Consider a hydrogen atom with nuclus at origo located above the surface of a perfect conducting metal at the position $\vec{d}$= (0,0,d) and an electron at  $\vec{r}$ =(x,y,z). The image charges induced in the metal will be
\begin{equation}
q=\frac{1-\epsilon}{1+\epsilon}e = -e, \epsilon \rightarrow \infty
\end{equation}
The potential due to this image interaction is then given by 
\begin{equation}
V_{W}(\vec{d},\vec{r}) \propto -\frac{e^{2}}{2d} - \frac{e^{2}}{2(d+z)} + 2\frac{e^2}{\mid 2\vec{d}+\vec{r} \mid}
\end{equation} 
where the two first terms are due to the interaction between the nuclus and it image and the electron and its image of itself. The third term is due to the two repulsive cross terms. Simple Taylor expansion in powers of $\frac{\vec{r}}{\vec{d}}$ leads to 
\begin{equation}
V_{W}(\vec{d},\vec{r}) \propto -\frac{e^{2}}{d^3}(\frac{x^{2}}{2} + \frac{y^{2}}{2} + z^{2}) 
\end{equation}
Thus the net result is an attractive potential which goes as
\begin{equation}
V_{W}(d) \propto \frac{-C_{vW}}{d^3}
\end{equation}
where C$_{vW}$ is the so-called van der Waals constant for the system which is dependent on the polarizability of the atom and the responce of the metal. Notice that the interaction does not require a permanent dipole and rare gas atoms may therefore also physisorb on the surface. An effect that we later shall utilize to determine the overall areas of surfaces.
\subsection{The repulsive part}
The attraction naturally does not continue for small d since the  electrons of the atom/molecule now will begin to interact strongly with the electrons of the surface. The kinetic energy of the solid increase as they will have to orthorgonalize to the localized electrons of the atom as a consequence of the Pauli repulsion. Some energy will natually be  gained as the same electrons also become attracted to the positive nucleus, but this can by no mean compenate the repulsion that occurs when it is not possible to make a chemical bonding as is the case for the rare gases. A simple approximation of the potential in this regime is an exponential function 
\begin{equation}
V_{R}(d) \propto e^{-\frac{d}{\alpha}}
\end{equation}
 since the desity of the electrons is decaying in this manner away  form the surface.
The resulting potential 
\begin{equation}
V(d) = C_{R}e^{-\frac{d}{\alpha}}-\frac{-C_{vW}}{d^3}
\end{equation}
is shown in figure 2. The depth of the well is usually of the order a few kJ/mole or less and the minimum will be several � from the surface. Thus molecules that are physisorbed on the surface are not chemically altered, but conserves theirs spectrosopic characteristica of the gas-phase. 

The effect is very similat to the molecular van der Waals interaction which makes gasses condense in multilayers. Here the potential have a very similar form as it is of the same nature. Often a  Lennard-Jones (N,6)-potential is used which has the form 
\begin{equation}
V(d) = \frac{-C_{n}}{d^n}-\frac{C_{6}}{d^6}
\end{equation}
and in particular the n=6 is popular for matematical reasons, despite the fact that an exponential description as above usually give a better description of the repulsive part\cite{Atkins}. The attractive part goes in this case as d$^{-6}$ \cite{Atkins}.

\section{Chemisorption}
\subsection{The Hydrogen molecule}
Before we treat chemisorption on surfaces i may be helpfull to first consider the most simple chemical bonding we can think of namely the bonding in the hydrogen molecule. This is not all that simpel and we will therefore start out with a H$_{2}^{+}$ ion. Consider to hydrogen atoms far away from each other then the electronic wave functions can be described by $\Psi_{a}$ and $\Psi_{b}$ which are eigenfunctions to the atoms with the energy $\epsilon_{a}$ and $\epsilon_{b}$. I we start to bring these two atoms close we know they will form a chemical bonding and end up as a molecule. The procedure for modelling this is to construct a new weavefunction from the atomic orbitals and the method is referred to as the linear combination of atomic orbitals (LCAO) medthod. In the following we shall neglect the spin. Thus the new wave function for one electron will have the form
\begin{equation}
\Psi=c_{1}\Psi_{a} + c_{2}\Psi_{b}
\end{equation}
and the Hamilton will have the form 
\begin{equation}
H = (-\frac{\hbar^{2}}{2m_{e}})\nabla^{2} +  (\frac{e^{2}}{4\pi\epsilon_{0}}\biggl( -\frac{1}{r_{a}} - \frac{1}{r_{b}}  + \frac{1}{R}\biggl) 
\end{equation}
where r${a}$ and r$_{b}$ are the distance of the single electron to the nuclei a and b.
By using the variation principle and solving the secular equations
\begin{equation}
\Sigma_{i} C_{i}({\bf H_{ik}} - {\bf ES_{ik}}) = 0,
\end{equation}
where H$_{ik}$ are matrix elements of the hamiltonian and S$_{ik}$ are the overlap matrix elements. Note that the atomic wavefunction are not orthorgonal.  These equations have  solutions when the determinat vanishes i.e when
\begin{equation}
\begin{vmatrix} \alpha-E & \beta -SE \\ \beta - ES & \alpha - E\end{vmatrix}
\end{equation}
where  we have used 
\begin{description}
\item[S$_{aa}$]= 1 since $\Psi_{a}$ was already normalized.
\item[S$_{bb}$]= 1 since $\Psi_{b}$ was already normalized.
\item[S$_{ab}$]= S the overlap intergral
\item[$\alpha$]= H$_{aa}$ = H$_{bb}$ = $\int \Psi_{a}((-\frac{\hbar^{2}}{2m_{e}}\nabla^{2}) +\frac{e^{2}}{4\pi\epsilon_{0}}\biggl(-\frac{1}{r_{a}} - \frac{1}{r_{b}} + \frac{1}{R}\biggl))\Psi_{a}dV$
\item[$\beta$]= H$_{ab}$ = $\int \Psi_{b}((-\frac{\hbar^{2}}{2m_{e}}\nabla^{2}) +\frac{e^{2}}{4\pi\epsilon_{0}}\biggl( -\frac{1}{r_{b}} - \frac{1}{r_{a}} + \frac{1}{R}\biggl))\Psi_{b}dV$
\end{description}
The result is two energy levels of the form:
\begin{equation}
c_{a}=c_{b} , c_{a}=\frac{1}{\sqrt{2(1+s)}}, E_{+}= \frac{\alpha+\beta}{1+s}
\end{equation}
\begin{equation}
c_{a}=-c_{b} , c_{a}=\frac{1}{\sqrt{2(1-s)}}, E_{+}= \frac{\alpha-\beta}{1-s}
\end{equation}
The terms $\alpha$ and $\beta$ can be rewritten renembering that the eigenvalue of the atomic hamiltonias is just E$_{1s}$ i.e.
\begin{equation}
\alpha = E_{1s} +\frac{e^{2}}{4\pi \epsilon_{0} R}-j_{1}
\end{equation}
where
\begin{equation} 
j_{1} = \frac{e^{2}}{4\pi\epsilon_{0}} \int \Psi_{a}\frac{1}{r_{b}}\Psi_{a}dV
\end{equation}
is a positive quantity. In a semilar manner can $\beta$ be rewritten as 
\begin{equation}
\beta = (E_{1s} +\frac{e^{2}}{4\pi \epsilon_{0} R})S-k_{1}
\end{equation}
where 
\begin{equation} 
k_{1} = \frac{e^{2}}{4\pi\epsilon_{0}} \int \Psi_{a}\frac{1}{r_{a}}\Psi_{b}dV
\end{equation}
In this manner can the two solutions be written as 
\begin{eqnarray}
E_{+} = E_{1s} +\frac{e^{2}}{4\pi \epsilon_{0} R}-\frac{j_{1}+k_{1}}{1+S}\\
E_{-} = E_{1s} +\frac{e^{2}}{4\pi \epsilon_{0} R}-\frac{j_{1}-k_{1}}{1-S}
\end{eqnarray}
Since both j$_{1}$ and k$_{1}$ are positive  will we get a  an upward shidt of both levels and then a upeward and down ward splitting as shown in Figure 3. Notice that the downward shift thus is smaller than the upward shift. The energy gain by forming the H$_{2}^{+}$ ion is thus the energy gain by moving one electron from E$_{1s}$ down indo the E$_{+}$ state.

In reality we wanted to consider the hydrogen molecule. The gross effect just corresponds to filling to electrons into the diagram show in figure 3 although the splitting will chage quantitatively.  For a hydrogen molecule will we have the hamiltonian:  
\begin{equation}
H = (-\frac{\hbar^{2}}{2m_{e}})(\nabla_{1}^{2} +\nabla_{2}^{2}) +  \frac{e^{2}}{4\pi\epsilon_{0}}\biggl( -\frac{1}{r_{1a}} - \frac{1}{r_{1b}} - \frac{1}{r_{2a}} - \frac{1}{r_{2b}} + \frac{1}{r_{12}} + \frac{1}{R}\biggl) 
\end{equation}
and we will here  use the valence band approch instead of the LCAO method to construt the two electron wave functions i.e.
\begin{equation}
\Psi=c_{1}\Psi_{a}(\vec{r_{1}})\Psi_{b}(\vec{r_{2}})+c_{2}\Psi_{b}(\vec{r_{1}})\Psi_{a}(\vec{r_{2}})
\end{equation}
agan neclegting the spin. The procedure follows the schemme give above for the ion except that a few new terms arises due to the changes in the hamiltonian.
The result of these consideration gives the eigen values:
\begin{eqnarray}
E_{+} = 2E_{1s} +\frac{e^{2}}{4\pi \epsilon_{0} R} + \frac{j_{2}-2j_{1}+k_{2}-2k_{1}S}{1+S^{2}}\\
E_{-} = 2E_{1s} +\frac{e^{2}}{4\pi \epsilon_{0} R} + \frac{j_{2}-2j_{1}-k_{2}+2k_{1}S}{1-S^{2}}
\end{eqnarray}
where 
\begin{eqnarray}
j_{2}=\frac{e^{2}}{4\pi\epsilon_{0}} \int \Psi_{a}^2(\vec{r_{1}})\frac{1}{r_{12}}\Psi_{b}^2(\vec{r_{2}})dV_{1}dV_{2}\\
k_{2}= \frac{e^{2}}{4\pi\epsilon_{0}} \int \Psi_{a}(\vec{r_{1}})\Psi_{b}(\vec{r_{1}})\frac{1}{r_{12}}\Psi_{a}(\vec{r_{2}})\Psi_{b}(\vec{r_{2}})dV_{1}dV_{2}
\end{eqnarray}
Usually the term J=j$_{2}$-2j$_{1}$ and K=k$_{2}$-k$_{1}$S are introduced as the coulomb and exchange integral respectively  and both are negative. The dissociation energy (2E$_{1s}$-E$_{+}$) is found to be 303 kJ/mol which is somewhat lower that the experimental value of 432kJ/mol. This can be improved by using other trial functions and introduce configuration interaction. The bottom line is that the two atoms gains energy by forming a molecule sharing the electrons where both electrons are in the bonding state. The other state is refereed to a the anti bonding state. It is also easily seen why for example two he atoms may not form a He$_{2}$ molecule. Adding to more electrons to the in principle same diagram means that we are also filling the anti-bonding orbital. Since the anti-bonding state is shifted more up in energy than the bonding state is shifted down will it scost energy to establish such a bonding and the interaction will be repulsive.

The simple picture presented above for the LCAO-MO method can be extended to more complicated cases. In the case of diatomic molecules is it worth noting that the atomic orbitals must have the same symmetry with respect to to rotations about the internuclear axis if the should have non-zero overlap. Two adjacent orbitals like (1s,1s), (1s,2s) or any (xs,ys) will have non-zero overlap S. Similarly will (p$_{z}$,p$_{z}$), (1s,p$_{z}$) and (p$_{x}$,p$_{x}$) have non-zero overlaps whereas combinations like (1s,p$_{x}$) or (p$_{x}$,p$_{z}$) will have no overlap. Thus it is simple to see which atomic levels will interact and split up in bonding and anti-bonding orbitals. It can also easily be seen that only orbitals which has similar energy will interact. If we consider the interaction considered earlier we ended up with a secular determinat of the form
\begin{equation}
\begin{vmatrix} \alpha_{a}-E & \beta -SE \\ \beta - ES & \alpha_{b} - E\end{vmatrix}
\end{equation}
where we now are considering two different levels interacting. Solving for E leads to 
\begin{equation}
E = \frac{\alpha_{a}+\alpha_{b}-2S\beta \pm \sqrt{(\alpha_{a}-\alpha_{b})^{2} -4(\alpha_{a}+\alpha_{b})S\beta+4\beta^{2}-4\alpha_{a}\alpha_{b}s^{2}}}{2(1-s^{2})} 
\end{equation}
Considering the limit where the overlap is small S$\rightarrow 0$  and the difference between the two levels $\delta = \alpha_{b}-\alpha_{a}$ is larger than the interaction $\beta$
\begin{eqnarray}
E_{+}  \approx \alpha_{a}- \frac{\beta^{2}}{\delta}+S\mid \beta \mid 
E_{-}  \approx \alpha_{b}+ \frac{\beta^{2}}{\delta}+S\mid \beta \mid 
\end{eqnarray}
Thus the splitting decreases with the energy difference between the interacting levels. Notice that there always is a repulsion term due to the fact that the elctrons have to orthorgonalize. With these simple rules in mind we can now set up themolecular orbital energy diagrams for some simple diatomic molecules and some examples of N$_{2}$ and O$_{2}$ are given in Figure 4. The stability of the molecules can be estimated from the number of bonding orbitals filled compared to the number of anti-bonding orbitals. I.e. it is eaily seen that  N$_{2}$ molecule is much more stable than the O$_{2}$ molecule since in the latter the two additional electrons are located in anti-bonding orbitals.
Naturally the the accurate nature and energy of these bondings can and have been investigated in much greater detail, but the simple considerations presented here captures the essential aspects of the chemical bonding we need in order to proceed towards the phenomena chemisorption we set out to investigate. 