\chapter{Depth Profiling with XPS and AES}
          In the previous chapters we have discussed the XPS  and  AES
          methods for surface analysis. Both methods are very useful
          for determination of the qualitative  composition,  although
          XPS clearly is the most advantageous for identifications  of
          chemical composition and to provide an accurate quantitative
          determination is called for. On the other hand, AES is
          advantageous if information of the lateral composition is
          necessary. Both methods can also be used for a determination
          deeper into the material. We have already seen that some
          depth information can be obtained by tilting the sample
          relative to the analyser. This, however, only gives a depth
          resolution of a few atomic layers. Therefore, the methods
          must be combined with other methods so that the interesting
          layer can be exposed to the surface sensitive analysis. This
          can be done mechanically by a broad variety of polishing
          procedures or by sputtering.

          \section{Instrumentation}

             The first class of methods  is  the  mechanical  type
          where the interesting part of  the  sample  can  be  exposed
          simply  by  making  a  cut  orthogonal  to  the  surface  or
          by ball cratering. In the first case the deep lying layers
          can then be analysed by use of SAM, see Figure 7.1. It is
          immediately seen that the resolution of this analysis will
          be limited by the resolution of the SAM available. This may
          not be trivial as it is not always easy to prepare a sample
          with a perfect cut orthogonal to the surface, especially not
          in the cases where the region is very inhomogeneous. The
          ball cratering method is very similar except that a crater
          is polished in the surface. Different layers will then be
          exposed and can again be analysed by use of SAM with
          sufficient resolution See Figure 7.2.

             In both cases the depth resolution is given by SAM (200
          � and upwards), although it can be improved  considerably  in
          the ball cratering method due to the fact that we here  have
          a cut under an angle. These methods are seldom used  because
          in general it is much more convenient to use  EDX  on  such
          samples.\\

          \vspace{1cm}

          \noindent {\bf Figure 7.1} Schematics of\\ SAM  analysis  of
          an\\ orthogonal cut.\\

          \vspace{6cm}

          \noindent  {\bf  Figure 7.2}  Schematics  of  \\  the  ball
          cratering method.\\

          \vspace{6cm}

          However, in the range 0-1000 � there is another  method  by
          which the surface can easily  be exposed and analysed  by  use
          of both AES or XPS,  namely  by  sputtering  with  rare  gas
          atoms, preferentially Argon. By ionising  Argon  at  a  high
          potential the ions created can be manipulated just like  the
          electrons and be focused into  a  spot  on  the  surface  and
          scanned if necessary. The formation  of  the  beam  is  very
          similar to the way we measured the pressure in  the  chamber
          as shown in Figure 2.4. Instead of collecting  the  ions  on
          the collector they are extracted from the cage formed by the
          grid and accelerated towards the  sample  which  is  usually
          kept at ground potential. In  this  manner  ion  beams  with
          energies 1-5 keV and with  currents  of  0-10$\mu$A  can  be
          formed. The Argon is supplied by leaking  in  Argon  in  the
            chamber  to   the   required   pressure   of   roughly
          1$\times$10$^{-5}$ mbar. This is  a  high  pressure  in  the
          context of UHV and might also cause problems for  some  types
          of pumps, like for instance the ion pumps, but as the gas  is
          inert it does not influence the surface analysis if the
          Argon is sufficiently clean. The pressure in the chamber can
          be reduced about three orders of magnitude if a
          differentially pumped ion gun is used instead. In this case
          a high pressure is still needed in the ionising region but
          the beam is then passing through a small hole into the main
          chamber. By pumping the ion-gun and the main chamber by
          separate pumps (differentially pumped) the pressure in the
          main chamber may be kept in the 10$^{-8}$ mbar region. The
          spot size diameter is, as for the electron beam, determined
          by the current density and will typically be of the order
          100 $\mu$m in standard equipment. The beam can be scanned
          to give a homogeneous exposure over a rather large area (10
          mm $\times$ 10 mm) of the surface so depth analysis with XPS
          can be carried out. For other applications, like for
          instance Secondary Ion Mass Spectroscopy (SIMS), the beam
          can be focused down to much smaller spots sizes. 




           \section{Ion Sputtering}


             Several types of  rare  gases  can  be  used  for  sputter
          profiling, but usually Argon is the preferred gas as  it  is
          cheap to purify and has a high sputter  yield.  When  a
          high energetic ion is hitting the surface the  ion  will  be
          neutralised and transfer energy to  the   surface  atoms
          through a number of essentially binary collisions.  Some  of
          the surface atoms will obtain sufficient energy and momentum
          so that they will be able to escape the surface,  see  Figure
          7.3. The atoms which in this manner are  scattered  off  the
          surface will by far leave as neutral species and  only  a
          small fraction will survive as ions. Since the ions essentially
          are coming from the top most layer and since they easily can be
          detected by a  mass  spectrometer  this  is  a  method  very
          useful for determining the composition into the surface  as
          the atoms are eroded away. This method  is  called  SIMS  as
          mention above and can be done in small spots. The method  is
          extremely sensitive (ppm level) but is rather  difficult  to
          quantify since the survival of the ions depend exponentially on
          the work function of the surface.  For  further  details  on
          this method we shall refer the reader to \cite{briggs2}. A
          method has been developed which overcomes this problem by
          detecting all the atoms coming off the surface.
          This is a much more reliable way to determine the surface
          depth composition. However, since this method relies on
          resonance ionisation by two lasers it is economically not
          viable.

             Therefore, it is much more convenient to simply use
          either XPS or preferentially AES to measure what is left on
          the surface after a certain number of atoms have been
          sputtered away by bombarding the surface with energetic
          Ar$^{+}$ ions. The measurements can be made simultaneously
          with sputtering or by alternating sputtering and
          measurements. The AES method is the preferred method because
          it is in general much faster than the XPS method. 

                      \vspace*{7cm}

          \noindent {\bf Figure 7.3}  Schematics  of  the  sputtering
          process.\\




             If we define a sputter yield Y as  the  number  of  atoms
          removed from the surface per incoming ion  it  is  possible,
          under ideal conditions,  to  convert  time  sputtered  to  a
          depth.

          The flux F of ions can be written as
          \begin{equation}
          F=\frac{I}{eA}
          \end{equation}
          where I is the ion current and A is the area exposed to the
          ion beam. In a pure element the number of atoms removed per
          second per area will be given by \begin{equation} g_{i} =
          FY_{i} \end{equation} g$_{i}$ can be converted to a depth by
          \begin{equation}
          \frac{dz_{i}}{dt}=\dot{z}_{i}=\frac{g_{i}}{N_{i}(z)}=\frac{FY_{i}(z)M_{i}}{\rho_{i}(z)
          N_{A}}=\frac{IY_{i}(z)M_{i}}{e A \rho_{i}(z) N_{A}}
          \end{equation} where N$_{i}$ is the density of atoms in the
          surface, M$_{i}$ is the mol weight, N$_{A}$ is Avogado's
          number , and $\rho_{i}$ is the density of this particular
          element i. If we assume that F, Y$_{i}$, and $\rho_{i}$ are
          constants with respect to time this differential equation
          can easily be solved and the result is \begin{equation}
          z(t)=\int_{0}^{t}\dot{z}dt = \frac{FYM}{\rho N_{A}} t =
          \frac{IYM}{e A \rho N_{A}} t \end{equation}
             This was a rather trivial example  since   it  only
          applies for pure elements. Let us instead look at a two  component  mixture  A
          and B where we want to determine the depth compositions
          (C$_{A}$(z) and C$_{A}$(z)). If we have two components,
          there will most likely be a difference in sputter yield for
          the two components leading to \begin{equation}
          z(t)=\int_{0}^{t}
          [C_{A}(t)\dot{z}_{A}+C_{B}(t)\dot{z}_{B}]dt \end{equation}
          Since C$_{A}$(z) and C$_{A}$(z) can be estimated by the
          surface analysis as a function of time and since $\dot{z}$
          can be estimated if the sputtering yield is known, it is
          possible to construct a plot showing the concentration
          profile of A and B into the material as a function of depth.
          Unfortunately it is generally not so simple since the
          sputtering yields for different compositions and compounds are
          not well known.


\subsection{The Sputter Yield}

             The essential parameter for doing  sputter  profiling  is
          the yield Y; the number of particles that is removed from the
          surface per incoming ion. The yield Y from a pure  amorphous
          target can formally be written as \begin{equation}  Y=\Delta
          F(E_{0}) \end{equation}  where  $\Delta$  contains  all  the
          material properties such as the surface  binding  energy  of
          the atoms sputtered and F(E$_{0}$) is the  energy  deposited
          at the surface depending upon the type of ions,  the  energy
          E$_{0}$, the incidence angle of  the  ions,  the  target
          atoms, and the density of the atoms in the target.

             The parameter $\Delta$  can  be  derived  describing  the
          number of target atoms that obtains  sufficient  energy  and
          momentum in the correct direction so that they can  overcome
          the barrier and escape the surface. The result is
          \begin{equation}
          \Delta \approx \frac{0.042}{NU_{0}}
          \end{equation}
          in units of  [� eV$^{-1}$] where N [�$^{-3}$] is the density  of  atoms  in
          the target and U$_{0}$ [eV] is the surface binding energy of
          the target atoms. For details  of  the  derivation  of  this
          expression we shall refer the reader to  \cite{sigmund}.  In
          Figure 7.4, the sputter yield  for  400  eV  Xe$^{+}$-ion  is
          plotted for most of the elements. It is  clearly  seen  that
          there is a rather strong variation. More relevant  data  can
          be  found  in  \cite{behrisch}  where  sputter  yields  for
          various ions, energies, and elements have been collected. All
          this works relatively well for the  pure  elements,  but  as
          soon as we  turn  to  inhomogeneous  samples  consisting  of
          alloys or compounds these sputter yields are in general  not
          valid any longer  since  for  example  the  surface  binding
          energy  changes.  Thus  there  is  a  general   problem   in
          converting the sputter time into a depth  for  more  complex
          samples.

          \newpage

\vspace*{11cm}

\noindent {\bf Figure 7.4} The sputter yield for 500 eV Xe ions.\\


          \section{Factors Limiting Depth Resolution  and  Accuracy
          of Profiles}


             It is straight forward to perform a sputter profile  and
          plot the  concentration  of  the  various  components  as  a
          function of sputter time or better  as  a  function  of  ion
          dosage. The problem arises when  we  want  to  interpretate
          these data in terms of depth and depth resolution.

          \subsection{The Effect of Attenuation}

          First of all, the AES-signals are not coming from  a  single
          layer but  are  exponentially  damped  signals  from  several
          layers. This means that we shall  never  expect  to  obtain
          extremely sharp features, although the sample studied consists
          of  such.  Consider  for example  a  thin   layer   of   Gold
          (d$_{Au}$=25 �) deposited on top of a Germanium  surface  and
          then covered by another layer of Germanium d$_{Ge}$=30 �. It
          is assumed that the deposition is ideal, i.e. it is  a  layer
          by layer growth mode, see Figure  6.15.  Using  the  equations
          6.10 and 6.11 we can easily  adapt  those  to  describe  the
          intensities from the constructed sandwich. The  signal  from
          the Gold will be given by
             \begin{equation} I_{Au}=
          I^{\infty}_{Au}(1-e^{-\frac{d_{Au}}{\lambda_{Au in
          Au}}})e^{-\frac{d_{Ge}}{\lambda_{Au in Ge}}} \end{equation}
          which is the Gold signal from  a  Gold  layer  of  thickness
          d$_{Au}$ damped  through  a  Germanium  layer  of  thickness
          d$_{Ge}$. The term $\lambda_{x in y}$  refers  to  the  mean
          free path of an Auger electron from element x moving through
          element y. Similarly, the signal from the Germanium  can  be
          written as
\begin{equation}
          I_{Ge}= I^{\infty}_{Ge}(1-e^{-\frac{d_{Ge}}{\lambda_{Ge
          in Ge}}})+ I^{\infty}_{Ge}e^{-\frac{d_{Ge}}{\lambda_{Ge
          in Ge}}-\frac{d_{Au}}{\lambda_{Ge in Au}}} \end{equation}
          where the first term is due to the Germanium  overlayer  and
          the second term is due to the Germanium substrate  which  is
          damped  both  through  the  Gold  layer  and  the  Germanium
          overlayer.

          This sandwich can now be analysed by sputter profiling. If
          it is assumed  that  the  sputter  profiling,  just  as  the
          deposition, is ideal i.e. removal layer by layer it is  easy
          to set up the equations describing the signals as a function
          of time. The sputter rate can be evaluated for each  of  the
          elements (eqn. 7.4) and the thickness of  the  various  layers
          can then be estimated as a function of  time  as  $d-\dot{z}
          t$. If it takes t$_{Ge}$  seconds  to  sputter  through  the
          Germanium overlayer and t$_{Au}$ for the Gold layer the time
          dependence of the signals  are  divide  naturally  into  two
          intervals. The  signal  for  the  Gold  layer  is  given  by
          \begin{equation}                 I_{Au}(t)                 =
          I^{\infty}_{Au}(1-e^{-\frac{d_{Au}}{\lambda_{Au           in
          Au}}})e^{-\frac{d_{Ge}-\dot{z}_{Ge} t}{\lambda_{Au in  Ge}}}
          \hspace{2cm} 0 \leq t \leq t_{Ge} \end{equation} and
          \begin{equation} I_{Au}(t) =
          I^{\infty}_{Au}(1-e^{-\frac{d_{Au}-\dot{z}_{Au} (t-t_{Ge})}{\lambda_{Au
          in Au}}}) \hspace{2cm} t_{Ge} \leq t \leq t_{Ge}+t_{Au}
          \end{equation}       The        result        for        the
          $(I_{Au}(t)/I_{Au}^{\infty})$ is plotted  vs.  sputter
          time for  a  value  of  $\lambda_{Au  in  Au}$  =  5  �  and
          $\lambda_{Au in Ge}$ = 10� is shown in Figure 7.5.  So  even
          in this very ideal case the profile of  the  Gold  layer  is
          broadened. Naturally this broadening can be corrected for as
          its origin is well understood. The above intensities are  all
          special          cases          of          \begin{equation}
          I_{i}=\frac{I^{\infty}_{i}}{\lambda_{i}}\int^{\infty}_{0}C_{i}(z')e^{-\frac{z'}{\lambda_{i}}}dz'
          \end{equation} where C$_{i}(z')$  is  the  concentration  of
          element i in depth z.

          Ideally we started out with a concentration profile
          C$_{i}(z')$ which in general is the unknown function we want
          to determine. The resulting intensity I$_{i}$(z) where z  is
          the sputter depth is a result of convolution of a resolution
          function $g(z-z')$ and the true concentration profile thus
\begin{eqnarray}
          I_{i}(z)
          &  =  &  \frac{I^{\infty}_{i}}{\lambda_{i}}C_{i}(z')*g(z-z')\\
          I_{i}(z)   &    =    &    \frac{I^{\infty}_{i}}{\lambda_{i}}
          \int^{\infty}_{-\infty}C_{i}(z')g(z-z')dz' \end{eqnarray}
          So in this case where
\begin{equation}
          g(z-z')=e^{\frac{z-z'}{\lambda_{i}}}dz
          \end{equation} is  well  established  there  exists
          routine for elimination of this broadening.  The  broadening
          in the profile can as a rule of thumb be estimated as $\Delta
          z_{lambda}= 1.6 \lambda$.

                            \vspace{1cm}

          \noindent {\bf Figure6.5} The relative\\ intensity  of  Gold
          from\\ the sandwich as a fuction\\ sputter depth.\\

          \vspace{8cm}



             \subsection{Other Effects}

             There are other effects leading both  to  broadening  and
          shifts of the true profile which are  not  as  easily  dealt
          with. The sputtering  process  itself  is  quite  a  complex
          process and we shall here briefly mention a  number  of  the
          effects that may lead to a broadening and introduce various
          errors when evaluating sputtering profiles.

           \subsubsection{Statistical Broadening}

             First of all    the  sputtering  process  is  a  random
          process, meaning that  we should never expect to  have  a
          removal of layer by layer. After some sputtering time we may
          have eroded 5 layers  off  at  one  place  of  the  surface,
          whereas 7 layers have been eroded away at other places.  This
          will lead  to  an  additional  broadening  which  is  usually
          considered to be proportional to the  square  root  of  the
          depth in the region below 100 �. For  higher  values  of  z,
          $\Delta$z appears to saturate at roughly 10-20 �.

          \subsubsection{Broadening  by  Surface  Roughness}

          Surface  roughness  will  naturally   also   influence   the
          resolution as will surface crystallography. Both effects can
          strongly influence  the  resolution  as  the  sputter  yield
          depends on the incoming angle of the ions. If for example  a
          polycrystalline sample is sputtered for  a  long  time,  the
          sputtering itself may introduce structural  changes  of  the
          surface leading to substantial broadening. This  is  clearly
          seen in Figure 7.6 where a polycrystalline Zinc sample has
          been sputtered to a depth of  3-4  $\mu$m  and  subsequently
          studied by electron microscopy.

           \vspace{11cm}

          \noindent {\bf Figure 7.6} Electron microscope picture of  a
          polycrystalline crustalline Zinc sample sputtered  to  3-4  $\mu$m  depth.\\

                          \newpage


          The surface which initially appeared flat is  now  extremely
          rough and  it  is  clear  that  any  concentration  profiles
          measured on such a surface will be completely  smeared  out.
          This problem can to some extent be eliminated  by  averaging
          the incidence angle of the ion beam by rotating  the  sample
          during the sputtering process. The effect is shown in Figure
          7.7 where a sputter profiling of a multi sandwich  of  Cr/Ni
          (each layer 500 � thick) has been analysed with and  without
          rotating the sample. The incident beam was Ar$^{+}$-ions  at
          1 keV at  68$^{\circ}$  angle  incidence  \cite{zalar}.  The
          depth resolution for the rotated sample is clearly  improved
          as seen in the lower panel of Figure 7.7.


                                  \vspace{11cm}

          \noindent {\bf Figure 7.7} Sputter profile of a multi  Ni/Cr
          sandwich without  (top  panel)  and  with  (lower  panel)
          simultaneous rotation of the sample.\\


          \subsubsection{Preferential Sputtering}

          If there is a large difference  in  sputter  yield  for  the
          various components in the sample the surface  concentration
          of the components may be changed due to the sputtering. Thus
          the surface composition will change from its true value to a
          composition determined by the steady state solution  of  the
          removal rates of the various components. This  leads  to  an
          enrichment of the component  which  has  the  lower  sputter
          yield and the sputter  profiling  will  reflect  a  too  high
          concentration of this particular component.

             \subsubsection{Recoil Mixing}
          The high energy  ions  hitting  the  surface  will  transfer
          energy and momentum to the surface atoms whereby there  will
          be a mixing of the atoms at the surface. This in itself leads
          to a broadening of the profiles, but furthermore there will
          also be a possibility that some atoms are recoiled into  the
          surface. This will result in a  shift  of  the  profiles  to
          higher depths as some of the atoms are recoiled further into
          the sample before they are removed.

          \subsubsection{Radiation Enhanced Diffusion and Segregation}
          These effects both lead to broadening  and  errors  in  the
          concentration  profiles.  Due  to  the   many   defects   the
          sputtering inevitably introduces in  the  surface  region  it
          will be much easier for the atoms to diffuse around. Thus if
          there is a component which has a very low surface energy, this
          component will prefer to segregate to the surface and  cause
          an  enrichment  here  which  does  not  reflect   the   true
          composition of the sample. Similarly,  any  sharp  interface
          will be smeared out by diffusion.

          Several of the above effects will lead to broadening and  as
          they  all  have  to  be  convoluted  it  is  appropriate  to
          approximate the resolution function by a Gaussian
          \begin{equation}
          g(z-z')=\frac{\pi                                     \Delta
          z^{2}}{2}e^{-2(\frac{(z'-z)}{\Delta z})^{2}}
          \end{equation}
          In general the resolution will decrease with depth as several
          of the discussed broadening effects are proportional to  the
          depth sputtered. For a more detailed and rigorous  treatment
          of the above discussed effects, the reader is referred to  an
          excellent review by Hoffman  \cite{hoffman}  and  references
          therein.


          \section{Practical Sputter Profiling}

          It is seen that  there  are  many  effects  which  make  an
          accurate interpretation of a  sputter  profile  a  difficult
          task, even in cases where sandwiches  of  pure  elements  are
          studied. Sputter profiles are therefore generally  presented
          as concentration profiles as a function of ion dosage.  This
          is often sufficient as it in many cases only is necessary to
          discuss qualitative differences between different samples. In
          the following we shall go through a few such examples.

          \subsection{Sputter Profile of Stainless Steel}

          In Figure 7.8 and Figure 7.9  the  sputter  profiles  of  a
          18-8 stainless steel sample are shown just after it had been
          cut off a rod and after a heat treatment at  600  C$^{\circ}$
          for  600  sec  respectively.

          \newpage

          \noindent {\bf Figure 7.8} Sputter profile of stainless steel.\\

          \vspace{11cm}
          \noindent {\bf Figure 7.9} Sputter profile of  heat  treated
          stainless steel.
          \newpage




            In   both   cases   a   carbon
          contamination  was  observed  at  the   surface   prior   to
          sputtering. Such a thin layer of  surface  contamination,
          is always observed on a sample that has been introduced into
          the apparatus  from  atmospheric  pressure.  It  is  readily
          removed after 10-20 sec of sputtering. Also other  forms  of
          contamination like sulphur and chlorine can be observed.  It
          is noticed that even on the non-treated surface (Figure 7.8)
          there is a substantial oxide  layer  and  an  enrichment  of
          Chromium on the cost of  Iron  and  Nickel  in  the  surface
          region. This enrichment is further developed when the sample
          was heated in atmospheric air at 600 $^{\circ}$ for 600  sec
          as shown in Figure 7.9. The surface layer can be  identified
          by XPS as  seen  in  Figure 7.10  and  consists  mainly  of
          Cr$_{2}$O$_{3}$ although less than 10 \% iron can  still  be
          observed.

          \vspace{11cm}

          \noindent {\bf Figure 7.10} XPS  spectra  of  the  stainless
          steel sample before and after sputtering.\\



          Approximately 500-1000 � must  be  removed  before
          bulk values of the components are observed. The fact that it
          is Chromium which is forming the protective oxide  layer  on
          the steel is in good agreement with rule of thump that it is
          always the most reactive component which will  segregate  to
          the surface region and be oxidised. The  reactivity  of  the
          transition metals is increasing from the right to the left.
          Therefore, if Titanium had been present in the steel we should
          have expected this to form an oxide overlayer as well. This
          simple picture holds well as long as the oxide layer grown
          is not too thick. For very thick layers (several $\mu$m)
          formed at high temperatures, the process becomes much more
          complex and NiO overlayers may be observed on top of the
          Cr$_{2}$O$_{3}$ \cite{alstrup}.




             \subsection{Investigations of Electrical Contacts}
          It has been mentioned  that  a  substantial  amount  of  all
          analysis performed by use of XPS and especially AES is  done
          within the field  of  microelectronics.  There  are  several
          problems which can be advantageously investigated with these
          methods and one of them is the formation of good contacts to
          the various semiconductors. Here we have chosen to show the
          contact  formation  to  Silicon.  Gold  usually  forms  good
          electrical contacts, but if it is deposited directly on  the
          Silicon it will readily diffuse into the Silicon and be  diluted.
          Therefore a layer of Chromium is deposited  on  the  Silicon
          prior to the Gold deposition. In  Figure 7.11a  is  the
          sputter profile of the contact shown as deposited. In  order
          to simulate ageing and use of  the  component  it  has  been
          heated to 300 C$^{\circ}$ for 2 hours and the result  is  shown
          in Figure 7.11b where it is clearly seen  that  the  Chromium
          layer itself is not a sufficient diffusion barrier since
          Silicon is observed in the Gold overlayer and visa versa.
          If the procedure is repeated but the Chromium is deposited
          in a background gas of Nitrogen some Chromium Nitride will
          be formed as shown in Figure 7.12a. Performing the same
          ageing experiment clearly shows, see Figure 7.13b, that this
          construction is much better since the mixing has been
          minimised. Thus, the Chromium nitride forms a diffusion
          barrier for Gold and Silicon.

          \subsection{Control of Thin Coatings}

          Coatings are used widely in industrial  processes  manly  for
          corrosion protection  or  for  chemical  inertness.  In  the
          latter case it is very common to deposit  a  thin  layer  of
          Gold on the surface. For example  many  metals  that  are
          used for bijouterie and the rim of glasses  will  contain
          Nickel. This is a highly non-desirable situation for  metals
          in contact with the human body  since  it  may  cause  allergic
          reactions. On the other hand,  Gold  is  a  rather  expensive
          material so the manufacturer wants to keep the layer as  thin
          as possible. Figure 7.13 shows a sputter profile of a sample
          where the manufacturer was cutting the expenses too low since he
          was not successful in avoiding  Nickel  at  the  surface.
          Only a few hundred � of Gold had been deposited and since
          Gold in this sort of industrial process is usually growing
          by an island growth mode, there will be areas free of Gold
          exposing the Nickel.

          \newpage

          \vspace*{9cm}

          \noindent {\bf Figure 7.11a} Sputter profile of Gold film  on
          Chromium on Silicon as deposited.\\

          \vspace*{9cm}

          \noindent {\bf Figure 7.11b} Sputter profile of Gold film  on
          Chromium on Silicon heat treated to 300 C$^{\circ}$  for  2
          hours.\\



                \newpage


          \vspace*{9cm}

          \noindent {\bf Figure 7.12a} Sputter profile of Gold film  on
          Chromium-nitride on Silicon as deposited.\\

          \vspace*{9cm}

          \noindent {\bf Figure 7.12b} Sputter profile of Gold film  on
          Chromium-nitride on Silicon heat treated to 300 C$^{\circ}$
          for 2 hours.\\


                           \newpage


          \vspace*{12cm}

             \noindent {\bf Figure 7.13} Sputter profile of  glass  rim
          consisting of Nickel coated with Gold.



               \newpage

\section{Problems}
             \begin{enumerate}





             \item In the semiconductor industry it is often a
          requirement to be able to make good electrical contact to
          silicon wafers. This is often done by evaporating a thin
          film of gold upon the silicon. The evaporation rate
          is, however, in our case not known. The rate is, therefore,
          estimated by  sputter profiling, that is measuring the
          surface composition while bombarding it with 2.0 keV
          Ar-ions. The ion fluency is measured to be 5
          $\mu$Acm$^{-2}$. It takes 22 minutes before the gold disappears
          and the silicon appears. This is idealised, how does it look
          in reality?. The sputter yield has earlier  been  determined
          to be 2.6 Au atom per Ar-ion. Determine the thickness of the
          gold overlayer. How  long  time  will  it  take  to  sputter
          through a silicon layer of  a  similar  thickness  when  the
          sputter   yield   for   Si    is    ca.    0.4    ?    Which
          assumptions/complications  must  be  considered  using  this
          method? Propose another and better method for  determination
          of film thickness in this thickness regime.




             \item Consider a sandwich structure consisting of a
          Si-substrate on which one monolayer of Ge (2.5 �) has
          been deposited and then 25 � Si on top. Finally the sandwich
          was terminated by another monolayer of Ge. All layers were
          grown epitaxiel i.e. layer by layer. In order to test the
          process the sample was controlled by sputter profiling using
          AES. The Ge LMM (1147 eV), MVV (52 eV) Auger lines and the
          Si LVV (92 eV) are measured as a function of sputter time.
          In the following we will assume that the sputter process is
          ideal and removes the overlayer layer by layer (discuss the
          complications of sputter profiling). Show how the
          intensities of the Ge lines as well as the Si line develops
          with sputter time. Give the relevant equations and discuss
          the approximations. Which of the two Ge lines would you use
          in order to get a good quality control? Are there ways to
          improve the profile? 









\end{enumerate}

