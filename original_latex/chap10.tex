\newpage
\chapter{Microscopy}
When investigating materials  and specifically catalysts it always of interest to get an impression for the three dimensional structure of the sample. This may be obtained through a strongly magnified picture of the material. Usually the conventional light microscopes are not supplying much help if we are going for the finer details such as nano-particles since  their resolution is limited by the wave length of the visible light to around 1$\mu$m. Thus if we are going for example nanometer or even atomic resolution we must turn to electron microscopes and scanning tunneling microscopes (STM). We have already in connection with scanning Auger seen that a scanning electron beam can make an excellent picture of the  three dimensional structure and that can be improved considerably by using several different methods for detection and by applying much higher energies (200-300 kV) of the electrons as it is done in dedicated electron microscopes. The ultimate resolution of electron microscopes allows a point resolution of about 1 �, sufficient for atomic resolution. The resolution is  strongly dependent on  the nature of the sample and it is in general not surface sensitive. The main emphasis here will, however, be put on the relative new  STM method. Compared to EM it is much less generally applicable, but on the other hand does  it offers a much more detailed view of the surface structure and dynamics, which have not been with in reach before. In STM is the  resolution  extremely good, less than 0.1 �, but it requires in general very flat  and well defined surfaces and is therefore in reality only applicable for model systems.

\section{The Scanning Probe Methods}

The idea of scanning probe microscopes is quite simple and comes now in several versions of which the most prominent are  the Scanning Tunneling Microscopy (STM)  and  Atomic Force Microscopes (AFM). The STM method was developed first by Binning and Roher   \cite{Roher1, Roher2, Roher3} (who got the Nobel prize for their invention) and this is also the most mature of the scanning probe methods see also \cite{STM}. Here a sharp tip (curvature of the order 100 �) is brought close to a surface and a low potential difference (the bias voltage) is put between sample and tip. Classically there should not be a current between the sample and tip, but as the distance gets small (5-10 �)  quantum mechanics takes over and the electron can now tunnel though the barrier. This leads to a current which can be measured and is typically set to be around 1 nA.  The experimental setup is shown schematically in Figure 10.1

\vspace*{11cm}

\noindent {\bf Figure 10.1} Schematic drawing of the STM setup.

\vspace{1cm}


The experiment can be performed in several modes i.e. constant current, or constant distance while scanning the tip over a desired surface area. The constant current mode is the most common. Here is the current kept constant by varying the tip-surface distance in a feed back loop. The tip is mounted on a piezo electric tube which can expand or contract dependent on the potential difference. Similarly is the scanning in the xy direction obtained by applying potential differences on a tube that is split in four parts. The coarse approach of the tip to the surface is taken care of by applying a so called inch worm also based on piezo electric elements. The scanning in the xy direction goes from � regime to a few  $\mu$m. The inch worm allows the tip to be move several mm per minute while the much more accurate z motion while scanning is taken care of by a tube that can expand or extract typically 1000 � with a typical resolution of 0.01� !!!. The hole arrangement is kept small to obtain high resonance frequencies and  mounted on a vibrational damped setup so external disturbances can be eliminated.  For obtaining a typical STM picture the tip is scanned 256 lines measuring in 256 point for each line. The scan rate is very fast and it takes between 1 and 10 seconds to obtain a picture. By acquiring subsequent picture it is possible to make movies of dynamical processes on the surface such as diffusion and chemical reactions. Usually a Tungsten tip is used. There are good recipes for preparing sharp tips which is necessary if the surfaces are not flat, but blunt tips prepared just by cutting a tungsten wire by a plier may also work since as long as just one atom is sticking further out than the rest this will conduct all the current as we shall see later.  The disadvantage of STM is  that it unfortunately only works on reasonable well defined flat conducting surfaces.

The other scanning probe microscopy method that should be mentioned in this context is AFM. The setup is in many ways similar to STM, although here the tip is mounted on a cantilever which is just a spring. When the tip is brought close to the surface will there be  van der Waals interaction due to coulomb interaction of quantum fluctuations in the tip and substrate. This interaction sets up a potential and the tip will first feel a force when it approaches the surface. This force can be measured by measuring how much the cantilever (spring) is bent knowing the force constant.  The forces involved are extremely weak (nano Newton) and very sensitive methods must be used to measure how much the cantilever is bent. This is typically done by shining laser light on the cantilever which is reflected into a position sensitive detector so the motion up and down can be measured. The hole arrangement is again mounted on a piezo electric scanner so it can be scanned over a desired area. In this manner it is possible to get atomic resolution on flat samples, or determining the  surface morphology on a larger scale. The advantage of this method is that it does not require a substrate that is conducting meaning that it can be used on real catalyst samples where the support material often is an oxide. The drawback is  that the shape of the tip, as for STM,  put rather strong restriction on the resolution  and the topography that can be depicted. Like STM  does the method not give any clue to the composition of the surface so this information must be obtained by using complementary methods like XPS. The AFM method still undergoes strong development and may be improved considerably  in the future.

In the following we shall concentrate on the STM since this is the superior method and where the most interesting research is performed at the moment.


\subsection{Theory of STM}

A relative simple and adequate theory for the tunneling effect seen in the STM was presented by Tersoff and Haman \cite{Tersoff}. Here we shall only give a brief review of the involved theory and derive an expression for the essential tunneling current. The experiment is shown schematically in Figure 10.2 where the energy levels of the electron in the tip and the sample are illustrated. Both are assumed to be free electron metals with monotonically density of states (DOS) around the Fermi level and work function $\phi_{\mu}$  for the tip and  $\phi_{\nu}$ for the crystal.  Figure 10.2 a illustrates the situation when the distance  between tip and surface is large. 

\vspace*{11cm}

\noindent {\bf Figure 10.2} Schematic drawing of the surface potential of the crystal and the tip.

\vspace{1cm}




The  distance in Figure 10.2b is now reduced to d and the tip is biased negatively with a few mV with respect to the crystal.
The potential for the electrons is reduced to a potential barrier with an average hight of $\phi = \frac{\phi_{\mu}   + \phi_{\nu}}{2}$ and as the wave functions for the electrons does not decay completely to zero they  may tunnel through the barrier. The expression for the tunnel current is in 1. order perturbation theory given by:
\begin{equation}
I=\frac{2\pi \mid e\mid }{\hbar}\sum_{\mu \nu}(f(E_\mu)(1-F(E_\mu)))\mid M_{\mu \nu} \mid  \delta (E_\mu + \mid e \mid V - E_\nu)
\end{equation}
where E$_{\mu}$ and E$_{\nu}$ are the energy relative to E$_F$ the Fermi level and
\begin{equation}
f(E) = \frac{1}{e^{\frac{(E-E_F)}{K_{B}T}} + 1}
\end{equation}
 is the Fermi function giving the occupation as a function of E. V is the applied voltage difference and if 
V$\mid e \mid \simeq $ 0 and T $\rightarrow$ 0 then f(E$_\mu$) $\simeq$ 1  for E$_\mu  < $ 0 and    f(E$_\nu$) $\simeq$ 0  for E$_\nu > 0$, thus electrons are tunneling from filled states in the  tip into the empty conduction states above the Fermi level of the  crystal. The delta function ensures energy conversation, i.e. it is assumed that the electrons do not undergo energy losses during the mechanism. This assumption is easily seen to be fulfilled  by  considering that typical bias voltages  are 10 mV and the universal curve discussed earlier. This will be truly ballistic electrons moving far into the crystal before undergoing energy losses. Using  V $\simeq$ 0 and the other  approximations we get 
\begin{equation}
I=\frac{2\pi  e^2 V }{\hbar}\sum_{\mu \nu}\mid M_{\mu \nu} \mid  \delta (E_\mu - E_F) \delta (E_\nu - E_F)
\end{equation}
The essential problem of calculating the tunneling matrix element $M_{\mu \nu}$ was solved by Bardeen \cite{Bardeen} giving
\begin{equation}
 M_{\mu \nu} = - \frac{\hbar^2}{2m} \int d\overrightarrow{S}(\Psi_ \nu^{\dag}\nabla\Psi_\mu-\Psi_\mu \nabla\Psi_\nu^{\dag}
\end{equation}
 Remembering  that i$\hbar \nabla$ is the moment operator it is seen that  this integral evaluates the flux of electrons though the  a surface S lying in the vacuum region.

In order to get any further we must introduce the appropriate wave functions. The wave functions at the surface of the crystal are straight forward  described by the two-dimensional Bloch expansion as:
\begin{equation}
\Psi_ \nu (\overrightarrow{r_\parallel},z) = \frac{1}{\sqrt{\Omega}} \sum _{G} a_G e^{i\overrightarrow{\kappa_G}\overrightarrow{r_\parallel}}e^{-z \sqrt{\kappa^2+\kappa_{G}^{2}}}
\end{equation}
where G is a surface reciprocal lattice vector,  $\Omega$ is a normalization volume, and  $\kappa$ is the decay constant of the surface wave into the vacuum region . $\kappa$ is given by
\begin{equation}
\kappa = \sqrt{\frac{2m\phi}{\hbar^2}}
\end{equation}
Thus the surface wave function is just an expansion on plane Bloch waves that are exponentially damped out in the vacuum region with the potential $\Phi$ as indicated in Figure 10.2. The evaluation of the tip wave functions $\Psi_\mu$ is much more difficult since its structure is not known in detail. One way to approach this problem is to assume that the end of the tip can be approximated by a sphere with radius R centered at position $\overrightarrow{r_0}$ as indicated in Figure 10.3. Then the tip waves can be approximated by s-waves of the type
\begin{equation}
\psi_\mu = \frac{1}{\sqrt{\Omega_t}}c_t \kappa R e^{\kappa R}\frac{e^{-\kappa \mid r-r_0 \mid}}{\kappa \mid r-r_0 \mid}
 \end{equation}
where now $\Omega_t$ is  the  normalization volume and c$_t$ is a normalization constant.

\vspace*{8cm}

\noindent {\bf Figure 10.3} Schematic drawing of the tunneling geometry.

\vspace{1cm}


The matrix element can now be calculated leading to 
\begin{equation}
M_{\nu\mu} = \frac{1}{\sqrt{\Omega_t}}\frac{4\pi \hbar^2}{2m}Re^{\kappa R}\Psi_\nu(\overrightarrow{r_0})
\end{equation}
inserting this in the equation for the current I we get
\begin{equation}
I = \frac{8\pi\hbar^3 e^2 R^2 V}{m}e^{\kappa R}D_t(E_F)\rho(\overrightarrow{r_0},E_f)
 \end{equation}
where D$_t  =  \frac{1}{\Omega_t}\sum_ {\mu} \delta(E_\mu-E)$ is the density of states per volume of the tip and
\begin{equation}
\rho(\overrightarrow{r},E_f) =\sum_{\nu} \mid\Psi_\mu \mid^2 \delta(E_\nu -E)
 \end{equation}
is the local density of states at the crystal . The surface wave functions are exponentially damped out in vacuum where the is a mean potential $\Phi$ resulting in
\begin{equation}
 \mid\Psi_\mu \mid^2  \propto e^{-2\kappa z}
 \end{equation}
leading to the final equation
\begin{equation}
 I   \propto e^{-2\kappa z} =e^{-1.025\sqrt{\Phi(eV)}z(�)}
 \end{equation}

The main result is that the current is exponentially dependent on the distance between the tip and the surface, mainly because the overlap between the empty and filled wave is dependent of the tails of the wave function out in the vacuum region. This explains why it is not extremely difficult to get a good and localized tip. Consider one atom on the tip that sticks out with just 1 � compared to the rest, then it will carry 90\% of the current. This also explains why it is possible to obtain atomic resolution in STM.
 Finally it should be mentioned that for  fixed tip conditions  we will probe the density of states (either empty if the tip is negatively biased or filled states if the tip is positively biased) of the crystal just below tip. Thus what is depicted in an STM experiment is {\bf not} the atoms but the density of states around the Fermi level. That this density varies around the atoms leading to the observed corrugation is not a surprise.

The above model involved many approximations and assumption, nevertheless, it captures the most important features of the STM experiment. Naturally there are regions where this picture will break down due to it simplicity. If for example the tip surface distance is getting very small the potential barrier will not be a simple average any longer and eventually there will be formed an contact. This contact can consist of a single atom and is then seen how the conductance becomes quantized. This is an interesting phenomena but beyond the scope of these notes. Also if the current is getting very high will this simple picture break down since we have completely neglected the interaction between the  tunneling electrons. Thus care must be exercised.



\subsection{Results from the Yellium Model}
In order to elucidate how adsorbates will be depicted in STM N. Lang \cite{Lang} undertook a number of investigations using  a simple model system. The tip was modeled by a Na atom adsorbed on a high electron density metal which can be described by the so called yellium model. Here the positive charge from the nucleies are smeared out and the free electron gas is solved on this positive background. One advantage of such models is that it is relative easy to map out the difference in eigenstate density introduced by adding an adsorbate to the yellium. Figure 10.4 shows such differences when Mo, Na, and S is adsorbed on a yellium. The calculation are only shown for m=0 component of the wavefunctions since it is these states that are mainly  responsible for  the tunneling current. The figure shows that Na introduces a resonance above the Fermi level while S introduces one below and all three components result in additional charge density at the Fermi level. 


Using this set up  and positioning the ``tip'' i.e. the yellium with an adsorbed  Na atom in front of another yellium it was shown that the current density was indeed localized to the Na atom as shown in Figure 10.5. The largest current density arrow is a factor of 25 bigger than the smallest current density j$_0$. The length and the thickness of the current arrows are proportional to 1+ln(j/j$_0$).
 
\vspace*{8cm}

\noindent {\bf Figure 10.4} Difference in eigenstate density between the yellium-adsorbate system and the clean yellium for different adsorbates as indicated, taken from \cite{Lang}

\vspace{1cm}  




\vspace*{8cm}

\noindent {\bf Figure 10.5}  Current density for a ``tip'' consisting of a Na atom adsorbed on a Yellium positioned in front of a clean yellium surface. The length and thickness of the arrows are proportional to 1+ln(j/j$_0$). Taken from \cite{Lang} 

\vspace{1cm}




If this ``tip'' is now used to scan over the another yellium where adsorbates like Na, S, and He have been adsorbed it is possible to estimate how much $\Delta$ s the tip have to be displaced vertically in order to keep the current constant when a small bias voltage is applied. This is depicted in Figure 10.6 showing that Na will appear as a much larger protrusion than S while He will be depicted as a depression in the yellium. This is in good agreement with Figure 10.4 showing that Na introduces a much higher electron density at the Fermi level than  S does and since the tunneling current was proportional to the density of states at the Fermi level it will lead to a larger retraction of the tip in the Na case. The He will naturally not adsorb at all, but will due to the Pauli principle exclude the electron leading to a lowing of density of states with respect to the clean surface. 

\vspace*{8cm}

\noindent {\bf Figure 10.6} Displacement of the ``tip'' in the  constant current mode for Na, S, and He adsorbed on yellium, taken from \cite{Lang}

\vspace{1cm}  


The important thing to be learned from this, is that adsorbates may not necessarily be depicted as protrusions, although they are geometrically. It all depends on the effective charge density at the Fermi level and it is well known that adsorbates like carbon, nitrogen, and  oxygen all are depicted as holes, since they are relative electronegative and removes charge density from the Fermi level. This also shows that we must know what sort of adsorbates we are dealing with. i.e. the experiment must be conducted under well defined conditions.

\section{Spectroscopy}
In all the considerations above the bias voltage was kept low so only the states around the Fermi level participated in the tunneling phenomenon. If we now relax this condition and used higher potentials V the tunneling current will be dependent on the density of states at higher energy levels, making it possible to map out DOS below and above the Fermi level doing spectroscopy. The  current can be approximated  by
\begin{equation}
I \propto \int_{E_F}^{E_F+\mid e \mid V} dE \rho_\nu(\overrightarrow{r_0,E}) \rho{}_\mu(E-\dim e \dim V)
\end{equation}
where $\rho_\mu$ is the total density of states of the tip and $\rho_\nu$ is the local density of the sample. The  $\rho_\nu$ can  be approximated by a product of the density at the surface times a penetration factor out into the forbidden vacuum region as:
\begin{equation}
 \rho_\nu(\overrightarrow{r_0,E}) \sim  \rho{}_\nu(E) e^{-\frac{2d\sqrt{2m(W-E+\frac{1}{2})\mid e \mid V}}{\hbar}}
\end{equation}
where W is the hight of the potential barrier.  This is intuitively easy to understand. The lower the energy E the broader and higher will the barrier be and the current will therefore be dampened exponentially. If we want to isolate $\rho{}_\nu(E) $ we must clean it from the influence of the penetration factor and $\rho_\mu$ . By measuring relations between I and V for a fixed distance and by plotting  (dI/dV)/(I/V) will any sharp feature in the surface (or tip) density of states become apparent. An example was given by Lang shown in Figure 10.7 where  the Na ``tip'' was used on a Ca atom adsorbed on yellium. The left panel shows the difference in density of states for the two adsorbates and the left panel shows the resulting (dI/dV)/(I/V) analysis. Only broad features are in general observed for metals so this method is, like UPS,  useless for elemental analysis.

\vspace*{9cm}

\noindent {\bf Figure 10.7} Left the introduced difference in density of states by adsorbing Na and Ca on yellium. Right the resulting (dI/dV)/(I/V) analysis, taken from \cite{Lang}

\vspace{1cm}  



It is though worth noticing that if we look at S adsorbed on the yellium we see that the appearance in a STM experiment will depend very much on the bias potential, since it, as shown in figure 10.8, leads to a reduced density of states at high voltages. Thus, as indicated in Figure 10.8 the S atom will appear as a protrusion for voltages below 1 volt, will vanish around 1.3 volt and become a depression for higher voltages. Thus the picture obtainable will be very dependent on the choosen bias voltage. This is in particular illustrated on semiconductors where there is a bandgap and dangling bond pairs to form occupied bonding states and empty anti bonding states at the surface. By making (dI/dV)/(I/V) analysis the various states can be found for a constant distance and by scanning  at well choosen voltages the different orbitals can be mapped out on the surface. 

\vspace*{11cm}

\noindent {\bf Figure 10.8} Left the introduced difference in density of states by adsorbing Na and S on yellium. Right the vertical displacement of the ``tip'' over the S atom as a function of bias voltage, taken from \cite{Lang}

\vspace{1cm}  

In the above we have assumed that the tip behaves metallic-like. Not much is in reality known for sure about the states of the tip and the ad atom actually conducting the current. Sometimes the tip changes in the middle of a scan and what earlier appeared as protrusions may suddenly in the middle of a frame be  depicted as depressions or visa versa. Such sudden and unexplained changes are often interpreted as tip changes, where the tip for example is picking up  an atom  from the surface changing its nature.


\section{Examples of STM Investigations}
STM has had a very strong impact on the field of surface science and STM have to some extent also revolutionized the structural investigation. Still it is not a trivial method to use like LEED where it is possible to obtain a LEED pattern within less than half an hour if the surface is well ordered.  It can be reasonably tricky to prepare the tip so that high resolution pictures can be  obtained. Sometimes, the STM  is working immediately, other times one may spend a day at the microscope and only occasionally see a structure. It is particularly difficult to obtain atomic resolution on metal surfaces since the corrugation here usually is less than 0.1 �.

The STM and LEED methods  complements each other in the sense that LEED rather fast gives an idea of the overall order on the surface, but not much information on the atomic level if that is not ordered. Here, on the other hand,  works the  STM very well. In principle  does STM  not require any order on the surface and domain boundaries and position of single atoms can be investigated in great detail. So in principle the STM is much more versatile, although it should be remembered, that just because we see an interesting structure within an area of 100 x 100 �$^2$, does it not need to be representative for the rest of the surface. Therefore several areas must  be investigated before any conclusions can be drawn. In the following shall we give a few  examples on the use of STM.

\subsection{Surface Structure}
Acquiring a STM picture of a surface structure actually provides us with a real space picture of the surface topography, although care should be exercised. Since  it is the electron density at the Fermi level we are mapping out atoms that sticks out from the surface may  not necessarily become depicted as protrusion. Usually it is possible to recognize the unit cell although that does not provide a solution to the structure. An example is shown in Figure 10.9 where the STM pictures and the LEED patter of the S-Cu(100) ($\sqrt{17} \times \sqrt{17}$)R14$^\circ$ is shown.

 Although it from  AES studies is known that this surface contain 8 sulphur atoms per unit cell only four of them can be identified as protrusions in the STM picture. The corresponding LEED pattern is quite complex as there are two domains. A model based on the STM picture have been proposed and is shown in Figure 10.10. 

\vspace*{9cm}

\noindent {\bf Figure 10.9} STM and LEED picture of the S-Cu(100) ($\sqrt{17} \times \sqrt{17}$)R14$^\circ$ structure, taken from \cite{Luigi}



\vspace*{9cm}

\noindent {\bf Figure 10.10} A schematic model of the S-Cu(100) ($\sqrt{17} \times \sqrt{17}$)R14$^\circ$ structure with 8 sulphur atoms per unit cell, taken from \cite{Luigi}

\vspace{1cm} 



 Sometimes it is possible to observe both regions of the clean metal and regions covered with an adsorbate so it is possible to extend the metal lattice into the adsorbate structure. This is demonstrated in Figure 10.11 taken from Besenbacher et al. \cite{Besenbacher1} where the STM picture of the Cu(110) and the missing row reconstructed O-Cu(110) (2x1) are shown. 

\vspace*{11cm}

\noindent {\bf Figure 10.11} 

\vspace{1cm} 


Also by detailed analysis of the surface as a function of adsorbate dosis is it possible to estimate how much of the substrate is being build into a new structure by following the mass transport of the substrate, i.e. following the  development of steps and holes made in the crystal as a function of dosis. By making movies of the oxidation of Cu(110) it  was  for example possible to suggest at structure for the high coverage structure O-Cu(110)c(6x2), which quite surprising, showed that Cu atoms were sitting highly uncoordinated on top of the previous formed missing row reconstructions as shown in figure 10.12 \cite{Besenbacher1}. A structure where the unit cell is very difficult to solve in the conventional LEED-IV analysis.

\vspace*{11cm}

\noindent {\bf Figure 10.12} 

\vspace{1cm} 

\subsection{ Two-Dimensional Alloys and Overlayers}
Since there is no help to be found from spectroscopic analysis for distinguishing two metals from each other it can be a difficult task to investigate metal overlayers and alloy formation. But again if  the development of the surface can be followed as a function of deposition time, changes can be related to the amount of metal added. It is in this manner possible to study growth and diffusion mechanisms even on homo epitaxial systems like Pt deposited on a Pt single crystal. Earlier, the  growth mechanisms on surfaces were classified into  three modes as  mentioned  in Chapter 6 (Frank-van der Merwe, Stranski-Krastanov, and Volmer-Weber growth modes). However, the STM have shown, that that was a much to simplified picture and that there is many other complex ways the growth process may proceed. Also alloy formation can be investigated and it was recently shown by use of STM that even metals which are known to be immiscible can form two-dimensional alloys in the surface. It has now been found to be a quite common phenomena \cite{Besenbacher2} and we shall here describe  one of the first systems that was found.

If Au is deposited on Ni(111) it was expected that the Au would grow epitaxial on the Ni surface because the system was known to be immiscible and the surface energy of Au is significantly smaller than Ni. The surprise was therefor big, when STM pictures, like the one shown in Figure 10.13, were measured \cite{Besenbacher3}. Since the coverage of the black hole increased with increasing Au dosis these were identified as being due to gold. At the same time also some very bright atoms appeared and it turned out to be Ni atoms pressed out from the surface by the gold. Thus a two-dimensional nearly random surface alloy was being formed in this case. Actually the Au atoms have been calculated to protrude by about  a quarter of an � from the Ni surface, but again due to the electron density around the Fermi level  they appears like depressions in the Ni lattice. This two-dimensional surface structure proved to be a very good way to control the  reactivity of Ni particles and a catalyst was later developed on that basis \cite{Science}.

\vspace*{11cm}

\noindent {\bf Figure 10.13} 

\vspace{1cm} 


An example of studying overlayers on the surface is the Co-Cu(111) system. Again the structures that was found when depositing Co at room temperature were not simple islands in the submonolayer regime. Instead,  two layer high islands were found surrounded by a single layer high island. By adsorbing CO it was possible to identify,  that the central island consisted of Co while the surrounding island consisted of Cu. An STM picture of CO on Co/Cu(111) at room temperature is shown in Figure 10.14 taken from \cite{Morten}. CO was only adsorbed on Co sites at this temperature, and the ($\sqrt{3}\times\sqrt{3}$)R30$^{\circ}$ of CO on Co can be recognized in the top layer \cite{Morten}. Around the apparent double layered Co island there is a Cu brim which with time will become double layered at room temperature. 

\vspace*{11cm}

\noindent {\bf Figure 10.14} 

\vspace{1cm} 



Based on such STM pictures with atomic resolution, a model was proposed which is schematically illustrated in Figure 10.15. Thus co-adsorption of for example CO made it possible to distinguish the Co from Cu. It should in this context be mentioned that it is usually very difficult to obtain pictures of CO adsorbed on surfaces, since these species diffuse very fast compared to the time scale of the STM experiment. Only in the saturation coverage regime will  the  CO structure  freeze out and pictures like the one shown in Figure 10.14 are  obtainable.


\vspace*{11cm}

\noindent {\bf Figure 10.15} 

\vspace{1cm} 


\subsection{Dynamics and Chemical Reactions}
By acquiring sequences of pictures it is possible to follow the dynamics on the surface if the processes are taking place on the same time scale as the measurements. This is mainly a matter of being able to control the temperature of the sample while scanning. Unfortunately, that is not a trivial problem, since the sample must be vibrationally isolated and any temperature difference may lead to thermal drift in the system. Nevertheless, there is a strong development in the field and there have recently been many studies both below and above room temperature. 

 For example can diffusion of ad atoms or adsorbates  be studied and by following their tow-dimensional movement  the diffusion constant can be extracted. Also regular chemical reaction may be followed. One example is the oxidation of methanol on the O-Cu(110) (2x1) structure where it has been shown that the reaction is very anisotropical and only takes place at the rim of the oxygen islands. This implies that at that particular temperature is the  mean field approximation not valid. Such experiment requires long time stability of the STM and surprisingly few good investigations have actually been performed. This is also an area of development, since there is a strong demand for being able to run the STM not only at elevated temperatures but also at elevated pressures, from the catalysis community. Any STM can run under ambient conditions, but the challenge is to get it running in atmospheres relevant to catalysis and study reactions at the surface. There have been attempts to obtain STM images of real catalysts, but sofar has it not been very successful. It is very difficult to  orient yourself on such non-ideal surfaces and find the active metal particles. Furthermore is there also the problem of the isolating support material. 





\section{Electron Microscopy}
Electron Microscopy is widely used giving us a visual impression of materials or biological samples in the regime where the light microscopes does not work any longer. Considering  the universal curve  it is obvious that these 100-400 keV electrons will penetrate way into the sample and the method is as such not a surface sensitive method, although it can be if combined with for example an Auger analyzer. There are many ways to detect the electrons used in the EM as shown in Figure 10.16 taken from \cite{Niemantsverdriet} and each of the methods adds complementary information on the sample  investigated.  

%\vspace*{11cm}

\noindent {\bf Figure 10.16} 

\vspace{1cm} 

The transmitted (TEM) or reflected (SEM) electrons may in the scanning mode be used for depicting the sample morphology. By measuring emitted x-rays (EDX), emitted electrons (Auger), or by measuring energy losses it is possible to make elemental identifications. The electrons may undergo diffraction, if there are favorable oriented crystalline particles so also crystallographic information may be obtained. An example  of a transmission electron micrograph, stimulating our imagination of a catalyst, is shown in Figure 10.17 where small  Rh particles are supported on a silica sphere  \cite{Datye}.


%\vspace*{11cm}

\noindent {\bf Figure 10.17} 

\vspace{1cm} 


 Another example is the highly undesired growth of carbon on Ni-catalyst as shown in Figure 10.18. Here it is seen how the carbon filaments grow out from the Ni-particles basically ruining the catalyst, taken from \cite{carbon}.

%\vspace*{11cm}

\noindent {\bf Figure 10.18} 

\vspace{1cm} 



EM microscopes usually comes in the conventional high vacuum regime (10$^-6$- 10$^{-7}$ mbar) or   in the UHV version (10$^{-10}$ mbar). But  recent development have made it possible to study samples under reasonable pressures (mbar regime) so catalysts can be studied under somewhat more realistic conditions following for example the dynamics as a function of gas composition. Care must naturally here be taken here since an 300 kV beam is not exactly innocent concerning introducing chemical reactions, that would not otherwise have been possible.





.
\newpage 
.
\newpage