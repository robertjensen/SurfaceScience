\chapter{The Auger Process}

             The Auger process was first  explained  by  Pierre  Auger
           in 1923 who was performing experiments  with  X-rays
          in  a  cloud  chamber.  He   observed   that   besides   the
          photoelectrons, which had an energy   dependent  on  the
          energy of the X-rays, there were also emitted electrons  with
          a constant energy and thus independent  on  which  source  had  been
          used. These so-called radiationless transitions were due  to
          the relaxation of the electron holes in the atom created  by
          the photoemission process. We shall from here on refer to
          these processes as Auger transitions.

          The process is illustrated in Figure 6.1.

            The  {\em  initial
          state} in this case consists of an atom where
          a hole has been created in a deep-lying core  level.  As  we
          shall see, it is not interesting how this initial  state  was
          created. This atom is now highly excited and will eventually
          undergo a relaxation process. In chapter 4 we saw that  this
          relaxation can lead to emission of an X-ray photon, however,
          as long as the binding energy of the initial  hole  is  well
          below 10 keV this process has a rather low  probability.  It
          is actually much more likely that a  weakly  bound  electron
          will drop down and fill the  initial  hole  and  instead  of
          emitting a photon use the energy to excite another  electron
          out of the atom, leading to the {\em final state}. As  shown
          in Figure 6.1 the final state consists of an  atom  with  two
          holes either in some core levels, or in the  valence  levels
          or any combination thereof.


             As for the photoemission process, the  kinetic  energy  of
          the emitted electron will simply be the  difference  between
          the initial state and the final state. A crude estimate will
          thus  be  given  by  the   Koopman   binding   energies   as
          \begin{equation} E_{kin}=E_{initial}-E_{final} \approx  E_{B
          initial}-E_{B final1}-E_{B  final  2}  \end{equation}  where
          E$_{x}$ refers to the one electron  binding  energy  of  the
          holes involved.

          Just as in the case of XPS  it  is  clear  that  the  energy
          distribution of the emitted electron will  provide  a  unique
          identification of which elements and how much of each is
          present in our sample. If the appropriate energy interval is
          chosen it will also be surface sensitive.  Thus  we  have  a
          method which is very  similar  to  XPS.  However,  there  are
          a number of differences  which  sometimes  makes  AES
          more useful than XPS or visa versa.


          \vspace*{14cm}

          \noindent {\bf Figure 6.1} Schematics of an  Auger  process.
          Notice that both initial and final states have two holes.\\


          \section{The Excitation Source}

          The major difference between AES and XPS is  the  manner  in
          which we get the atoms to emit electrons.  We  have  earlier
          seen (see Figure 4.7) that  there  will  always  be  emitted
          Auger electrons as a result of the relaxation of  the  holes
          created by the photoemission process. This is however a very
          inconvenient way of generating the necessary  initial  holes
          for the Auger process. It is much more  useful  to
          use a high energetic electron beam typically from 3-10  keV.
          As we saw under the discussion of the  various  energy  loss
          mechanisms,  energetic  electrons   will   have   a   finite
          possibility of ionising the  atoms  in  the  substrate  and
          hereby creating the necessary initial holes in a deep-lying
          core level (see Figure 2.10).  The  advantage  of  using  an
          electron beam is that it  can  be  focused  and  manipulated
          rather well and still have a high current density.

          The electron beam is generated by use  of  an  electron  gun
          which is usually mounted inside  the  CMA  as  indicated  in
          Figure 3.1. By passing current through a filament it will  be
          heated sufficiently so that electrons can evaporate  and  be
          accelerated towards the sample. A number  of  lenses  ensure
          that the beam diameter  can  be  controlled  and  deflection
          plates at the end of the gun ensures that the  beam  can  be
          manipulated to any desired position on  the  surface  or  be
          scanned, see Figure 6.2.\\

          \vspace{12cm}

          \noindent {\bf Figure 6.2} Schematics of an electron gun for
          AES.\\

          As the lens system in the electron gun is  an  electrostatic
          imaging of  the  electron  source  onto  the  sample  it  is
          advantageous to  use  a  LaB$_{6}$  source.  By  mounting  a
          LaB$_{6}$ crystal on a filament a very good point source can
          be obtained and a  spot  diameter  down  to  200  ANGSTROM  can  be
          achieved with a current of  0.1  nA.  Actually  even  higher
          resolutions can be obtained as in the  electron  microscopes,
          but here we must remember that  a  sufficiently  high  current
          density is required if we shall obtain Auger spectra with a
          reasonable signal to noise ratio. As this  demands  no  less
          than 0.1 nA and as it is not possible to  confine  electrons
          to spots smaller than 200 ANGSTROM at such high  current  densities
          (unless going to much higher energies which then causes other
          problems)  this  in  reality  sets  the  limit  for  lateral
          resolution in AES.



          \section{The Fine Structure in AES}






          As mentioned the kinetic energy of the Auger  electrons  is
          given by the difference between the initial and final state.
          Thus  eq. 6.1 is only very crude as we know that the atom
          will undergo relaxation's when  the  initial  hole  is  being
          created. A better approximation  is  therefore  to  take  the
          binding energy of the two final holes to be  an  average  of
          the binding energy for the atom considered Z and the binding
          energy it would have in  an  atom  with  Z+1.  The  kinetic
          energy could then be approximated by
\begin{equation}
          E_{kin}  \approx  E_{B  initial}^{Z}  -
          (E_{B    final1}^{Z+1}+E_{B    final    1}^{Z}     +     E_{B
          final2}^{Z+1}+E_{B final 2}^{Z})/2
           \end{equation} where E$_{x}^{Z}$
          refers to the  one  electron  binding  energy  of  the  hole
          x in atom Z.
          In Figure 6.3 is the kinetic energy for the strongest Auger
          lines given for most of the elements. It is obvious that  it
          is nearly always possible to find  an  Auger  line  with  an
          energy in the interval where the electrons will  be  surface
                         sensitive.\\




          \subsection{Nomenclature}
          An Auger transition is always  named  XYZ  describing  which
          energy levels are participating  in  the  transition.  The  X
          refers to the initial hole, whereas the Y and  Z  refers  to
          the two holes in the final state. In order to make this more
          complicated than necessary the old X-ray  notation  is  used
          for these levels. Here the various levels have the  following
          names:

\vspace{0.5cm}

             \noindent   1s   \hfill   K\\   2s,   2p$_{\frac{1}{2}}$,
          2p$_{\frac{3}{2}}$ \hfill L$_{1}$,  L$_{2}$,  L$_{3}$\\  3s,
          3p$_{\frac{1}{2}}$, 3p$_{\frac{3}{2}}$,  3d$_{\frac{3}{2}}$,
          3d$_{\frac{5}{2}}$   \hfill   M$_{1}$,   M$_{2}$,   M$_{3}$,
          M$_{4}$,       M$_{5}$\\       4s,       4p$_{\frac{1}{2}}$,
          4p$_{\frac{3}{2}}$, 4d$_{\frac{3}{2}}$,  4d$_{\frac{5}{2}}$,
          4f$_{\frac{5}{2}}$,   4f$_{\frac{7}{2}}$   \hfill   N$_{1}$,
          N$_{2}$,  N$_{3}$,  N$_{4}$,  N$_{5}$,  N$_{6}$,   N$_{7}$\\


         \newpage
          \vspace*{14cm}

          \noindent {\bf Figure 6.3}  Energies  of  the  strongest
          Auger transitions.


          \subsection{Mulitiplet Splitting in AES}

             If we consider a transition starting with  a  hole  in
          the 1s shell which  results in two holes in  the  2s
          or 2p shell then the name of the  transition  would  be  KLL
          which  is  the  dominant  type  of  lines  for  the  lighter
          elements. This type of transition can naturally  be  divided
          up  into   different   combinations   like:   KL$_{1}$L$_{1}$,
          KL$_{1}$L$_{2}$,      KL$_{1}$L$_{3}$,      KL$_{2}$L$_{3}$,
          KL$_{2}$L$_{2}$, KL$_{3}$L$_{3}$ leading to six lines.

             In Figure 6.4  this  type  of  transition  is  shown  for
          Magnesium and it is easily seen that the spectrum cannot  be
          explained by only four lines. The reason that  we  only  got
          six lines is that we used a notation  only  appropriate  in
          jj-coupling.  We  saw  earlier  that  when  more  than   one
          localised hole is present it  is  more  appropriate  to  use
          intermediate coupling starting either from jj-coupling or LS
          coupling.\\

        \noindent {\bf Figure 6.4} A KLL Auger\\ spectrum of Magnesium\\

             \vspace*{9cm}


             Thus  in  the  intermediate  coupling  scheme  the  above
          transitions would lead to the following final states:

             \vspace{0.5cm}

          \noindent     KL$_{1}$L$_{1}$      \hfill      $^{1}S_{0}$\\
          KL$_{1}$L$_{2}$,           KL$_{1}$L$_{3}$            \hfill
          $^{1}P_{1}$,$^{3}P_{0}$,$^{3}P_{1}$,$^{3}P_{2}$\\
          KL$_{2}$L$_{3}$,  KL$_{2}$L$_{2}$,  KL$_{3}$L$_{3}$   \hfill
          $^{1}S_{0}$,$^{3}P_{0}$,$^{3}P_{1}$,$^{3}P_{2}$,$^{1}D_{2}$\\

          \vspace{0.5cm}

             Thus the simplest transition we can think of consists of a
          substantial number of lines. If the initial state is not  an
          s-electron there will also be a  splitting  of  the  initial
          state due to a spin orbit splitting doubling the  number  of
          Auger lines. An example of this type of transition is  shown
          in  Figure 6.5  where  the   M$_{45}$N$_{67}$N$_{67}$   for
          Ytterbium and Gold are shown. The  spin-orbit  splitting  of
          the N$_{45}$ level is very large in both  cases  leading  to
          two well-separated multiples. A theoretical description  of
          the line shape (using the intermediate coupling  scheme)  is
          possible for the Gold, but not for the Ytterbium where other
          types of Auger transitions obscure the comparison. Due to an
          M$_{4}$M$_{5}$N$_{67}$ there will be a  transition  starting
          with a hole both in  the  M$_{5}$  and  the  N$_{67}$  level
          leading to  a  multiplet  structure  right  on  top  of  the
          M$_{5}$N$_{67}$N$_{67}$ lines. The spectra shown  in  Figure
          6.5 was actually induced  by  the  continuos  bremsstrahlung
          emitted from an Aluminium anode.\\

          \vspace*{15cm}
          \noindent {\bf Figure 6.5} The experimental Auger spectra of
          Yb and Au and the theoretical fits.\\


          So far we have not discussed the transition  probability  for
          the Auger transition. As the relaxation process  is  due  to
          the Coulomb interaction between the involved  electrons,  the
          interaction Hamiltonian  will  be  described  by  a  Coulomb
          interaction and the transition  probability  will  be  given
          again  by  Fermi's  Golden  rule.   In   the   frozen   core
          approximation and one-electron picture this results in \cite{wentzel}:

          \begin{equation}
           P_{Auger}=\frac{4\pi^{2}}{h}|<\Phi_{Initial}|\frac{e^{2}}{r_{12}}|\Phi_{Final}>|^{2}
          \end{equation}
          where initial and final states refers to the  two  vacancies
          in the initial and final state.  This  expression  has  been
          used to  calculate  the  probability  for  going  from  each
          possible initial state to any available final state as shown
          in Figure 6.5. In general the  multiplet splitting  of
          Auger transitions are quite complex leading to rather  broad
          features as we have already seen.  The  lifetime  broadening
          will contribute significantly as there  are now three  holes
          involved in the process. Furthermore, just as was  the  case
          for the photoemission process, the  Auger  process  will  be
          followed by various shake-up and shake-off  processes  leading
          to satellite structures. They are therefore  not  as  useful
          for chemical identification and investigations  of  electron
          structures as the XPS lines. But if we disregard the  details
          of the structure and simply use the Auger lines as a  finger
          print of the various elements they can be very useful.











               \section{AES for Qualitative Analysis}


          When creating a deep lying  hole  there  are  two  competing
          possibilities for relaxation of the excited atom.  One  is
          the emission of a photon with the  probability  $\omega_{X}$,
          which was very useful for generation of X-ray  for  XPS,  and
          the other is a decay trough emission of  an  Auger  electron
          with the Probability $\omega_{Auger}$. The two channels  are
          competing, but in  general  $\omega_{Auger}$  will  be  much
          larger than $\omega_{X}$  for  initial  holes  with  binding
          energy less than 10 keV. See for example  Figure  5.6  where
          the probability of the two types of  relaxation  is  plotted
          for a hole in the 1s shell  as  a  function  of  the  atomic
          number. The binding energy of the 1s level in  element  Z=33
          (As) is 11.8 keV. Thus for electron beams with  energy  less
          than 10 keV the Auger process will prevail.\\

          \vspace{1cm}

          \noindent {\bf Figure 6.6} $\omega_{X}$ and  $\omega_{A}$\\  as  a
          function of Z\\ for the 1s shell.\\

          \vspace{4cm}

          The emission of the  X-rays  actually  has  a  very  useful
          aspect since they also carry information on  which  elements
          are present in the sample. This method  is  known  by  the
          name  EDX  and  is  mainly  performed  in  combination  with
          electron microscopy. It is  complementary  to  the  methods
          discussed here as  it  is  not  surface  sensitive  (usually
          probing $\approx 1 \mu$m) on the atomic scale. Besides it is
          not a very useful method for detection of the light elements
          (Z<11) as most of the relaxation in this low binding  energy
          regime takes place through Auger transitions.


          An Auger  spectrum  of  Silver  excited  by  use  of  2  keV
          electrons is shown in  Figure  6.7.  The  N(E)  spectrum  is
          completely dominated of the elastic  scattered  electrons
          and primary electrons which have undergone energy  losses  in
          the solid.\\

          \vspace{9.5cm}

          \noindent {\bf Figure 6.7} The raw N(E)  spectra  of  Silver
          excited by 2 keV electrons.\\

          Weak structures can be observed just below the  elastic  peak
          (energy loss features of Silver) and  around  350  eV.  This
          region has also been shown as N(E)x10 and it is  now  easily
          seen that there is a structure here.  By  consulting  Figure
          6.3  these  lines   can   be   identified   as   the   Silver
          M$_{45}$N$_{45}$N$_{45}$ transition. The lines  a  lying  on
          top of the high continuos background of secondary  electrons
          which basically carries no information and it  is  therefore
          convenient to differentiate the spectrum as shown in the top
          panel. The nearly constant background is then eliminated and
          the  structure  of  the  line  is  easily  recognised.   The
          intensity of the lines, assuming a Gaussian line shape,  can
          then be approximated by the peak to peak height. Just as for
          XPS  all  the  Auger  spectra  of  the  elements  have  been
          collected in a Handbook \cite{handbook} and Figure  6.8  and
          6.9 show a few examples of  such  standard  Auger  spectra.
          \newpage

          \vspace*{10.5cm}

                  \noindent {\bf Figure 6.8} The standard  Auger  spectrum  of
          Aluminium.\\


             Figure 6.8 shows the Auger spectrum of a  relative  clean
          Aluminium surface. Apart  from  several  low  level
          contamination features of Ar, C, and O, two main features from
          Aluminium are observed. The low lying line at 68 eV  is  due
          to an L$_{23}$VV transition. The VV refers to the  fact  that
          the two final states are located in the valence band and the
          shape of the Auger line in this case will actually be a self
          convolution of the valence band. The high energetic line  at
          1396 eV is due to a KL$_{23}$L$_{23}$  transition  which  is
          much weaker. The features below 1390 eV can be accounted for
          by energy losses to plasmons. If this surface is oxidised to
          Al$_{2}$O$_{3}$ the spectrum undergoes rather strong changes
          as shown in Figure 6.9. Both transition shifts downwards  in
          energy by 17-18 eV and  the  energy  loss  features  change
          dramatically. This  is  a  result  of  chemical  shifts  and
          changes in the relaxation mechanisms. The Aluminium  has  by
          oxidation been changed from a  very  good  conductor  to  an
          isolator. Thus the valence electrons have been shifted down in
          energy and there are no delocalised electrons around to
          effectively screen the excited final state. Therefore, the
          characteristic plasmon energy losses are no longer observed.
          In this case it is possible to identify the chemical
          state of the Aluminium, especially when the presence of
          oxygen at 510 eV is taken into account. \newpage 

           \vspace*{10cm}

          \noindent {\bf Figure 6.9} The standard  Auger  spectrum  of
          Al$_{2}$O$_{3}$.\\


             Similar chemical identification can be done for carbon as
          seen in Figure 6.10. Here the carbon Auger feature is depicted
          for a number  of  carbon  containing  compounds.  It  is  in
          general possible to differentiate between carbon in covalent
          bonding and ionic bonding. The carbides display a  spectrum
          characteristic of a filled shell structure  like  Neon.  The
          same features are observed  for  Nitrogen  in  nitrides  and
          naturally for Oxygen and Fluorine. Figure 6.11 shows  an
          AES study of the decomposition of methylamine on Nickel. At
          very low temperatures the methylamine is chemisorbed
          associatively on the surface and the spectrum reflects to a
          high degree the line shape expected from molecules. As the
          surface is heated the molecule will decompose and finally
          all the hydrogen will desorb and the Nitrogen and Carbon
          will react with the surface forming carbides and nitrides
          which can easily be identified by their characteristic line
          shapes \cite{chorkendorff3}.

          Thus chemical identification  can  sometimes  be  undertaken
          with AES, but  certainly  not  as  efficiently  as  with  XPS.
          Besides there are other problems with AES  which  should  be
          mentioned when discussing chemical identification.  Since  a
          high energetic electron beam is used with  a  relatively  high
          current density, there is always a potential danger that  the
          excitation source itself introduces chemistry. This is not  a
          severe problem for metals and alloys, but it is serious for
          chemical compounds like polymers or the methylamine shown
          above.

             \newpage \noindent {\bf Figure 6.10} The auger line
          shape\\ of the KLL transition in various\\ chemical
          surroundings. 

                                 \vspace{9cm}

             \noindent  {\bf  Figure 6.11}  The  Auger  spectrum \\  of
          methylamine adsorbed on Ni(111)\\ during decomposition.

                                           \newpage



           If the AES spectrum of the chemisorbed  methylamine
          shown in the bottom of Figure 6.10 was  repeated  after  the
          surface has been exposed to the electron beam for 5 min  the
          spectrum would look exactly as if it had reacted all the way
          to Nickel-nitride and -carbide. Thus great care  has  to  be
          taken when AES is used on compounds.


             \section{AES for Quantitative Analysis}

             Carrying out quantitative analysis with AES is very
          similar to XPS, although there are few complicating factors.
          The probability for getting an Auger electron XYZ out from a
          thin layer of element x in the depth z will be given by
          \begin{equation} dI^{XYZ}_{x}(z) = N_{x}\sigma_{X}(z)
          \omega_{XYZ} e^{-\frac{z}{\lambda_{XYZ}}} i(z) T(E_{XYZ}) dz
          \end{equation} where N$_{x}$ is the number of
          atoms$\times$cm$^{-3}$, $\sigma_{X}(z)$ is the cross
          section for generating the initial hole X in the depth z,
          $\omega_{XYZ}$ is the probability that the initial hole will
          decay as an Auger process, ${\lambda_{XYZ}}$ is the
          inelastic mean free path of the Auger electron, T(E$_{XYZ}$)
          is the probability that the electron will be detected by the
          analyser, i(z) is the flux of primary electron in the depth
          z. The electron flux of primary electrons vary with the
          depth because these electrons will undergo energy losses and
          because they will scatter elastically in the solid. 

             If we  consider  a  pure  element  and  assume  that  the
          dependency on depth is weak all these various factors can be
          put  together  in  one  constant   namely
            \begin{equation}
          I^{\infty  XYZ}_{x}   =   N_{x}\sigma_{X}   \omega_{XYZ}   i_{0}
          T(E_{XYZ})\lambda_{XYZ} \end{equation} which is the  signal
          we will obtain from the clean sample.
             Equation 6.3 would then read
            \begin{equation}
          dI_{x}^{XYZ}(z)  =   \frac{I^{\infty   XYZ}_{x}}{\lambda_{XYZ}}
          e^{-\frac{z}{\lambda_{XYZ}}}dz
          \end{equation}
             By integration from zero depth to infinity we get
               \begin{equation}
          I^{\infty   XYZ}_{x}   =   \int   dI_{x}^{XYZ}(z)    =
          \int_{0}^{\infty}\frac{I^{\infty     XYZ}_{x}}{\lambda_{XYZ}}
          e^{-\frac{z}{\lambda_{XYZ}}}dz \end{equation}
          By measuring  the  signal  from  the  pure  elements  it  is
          possible to establish a set of sensitivity factors just as it
          was done for the XPS lines.
               \begin{equation}
          I^{XYZ}_{x} = S_{x}^{XYZ} N_{x}
          \end{equation}
             whereby the composition of a sample can be estimated as
                  \begin{equation}
           C_{x}                                                      =
          \frac{\frac{I^{XYZ}_{x}}{S_{x}^{XYZ}}}{\sum_{i}^{N}\frac{I^{XYZ}_{i}}{S_{i}^{XYZ}}}*100\%
            \end{equation}
          where the sum i goes over all N elements present in the sample.

             The  relative  sensitivity  factors  for  most   of   the
          strongest Auger lines are shown in Figure 6.12 for a primary
          electron beam of 3 keV and in Figure 6.13 for a primary beam
          of 10 keV. Both Figures are taken from \cite{handbook}.

          \vspace{12cm}

             \noindent {\bf Figure 6.12} The sensitivity  factors  for
          Auger transitions excited by 3 keV electrons.\\

          Several  approximations  were   necessary   to   achieve   this
          simple expression and the compositions determined  in  this
          manner will therefore also have a considerable  uncertainty.
          Thus  for  investigations  where  accurate  determination  of
          the surface composition is mandatory  XPS  should  be  used.
          However, the  AES  method  is  very  useful  especially  for
          determining relative compositions between different areas of
          the samples or  depth analysis.
          \newpage

                  \vspace*{12cm}

             \noindent {\bf Figure 6.13} The sensitivity  factors  for
          Auger transitions excited by 10 keV electrons.

          \subsection{Example}
          In this example we shall demonstrate how the attenuation  of
          the Auger signals can be used to study growth  mechanisms  on
          surfaces and how it can be used to determine  the  inelastic
          mean free path.  Figure 6.14 shows the  natural  logarithm  of
          the Auger signal for two Germanium  lines  relative  to  the
          pure Germanium   as  a  function  of  Silicon  coverage
          deposited on top of the  Germanium.  It  is  seen  that  the
          Germanium signal is damped exponentially. If we assume  that
          the analyser was positioned normal to  the  surface  we  can
          estimate $\lambda$ for the two different  kinetics  energies
          if the coverage of the Silicon is known  and  if  we  assume
          that it is deposited layer by layer.
          The Germanium intensity will be given by
          \begin{equation}
          I^{XYZ}_{Ge}(d_{Si})=I^{\infty XYZ}_{Ge}e^{-\frac{d_{Si}}{\lambda_{Si}}}
          \end{equation}
          where $d_{Si}$  is  the  thickness  of  the  overlayer.  The
          Germanium signal will be damped through the Silicon and it is
          therefore $\lambda_{Si}$ which is determined in this manner.

          \noindent {\bf Figure 6.14} Attenuation of the\\ Ge LMM  (1147
          eV) and Ge MVV\\ (52eV)  Auger  lines  as  a\\  function  of  Si
          Coverage.\\

                      \vspace{9cm}

          By plotting ln($\frac{I^{XYZ}_{Ge}(d_{Si})}{I^{\infty    XYZ}_{Ge}}$)
          vs. d$_{Si}$,  $\lambda_{Si}$  can  be  found  for  the  two
          energies.

             Similarly the signal of the  Silicon  Auger lines  can  be
          followed and the intensity of the Silicon Auger lines  would
          have the following form as a function of  thickness
          \begin{equation}
          I^{XYZ}_{Si}(d_{Si})=\int_{0}^{d_{Si}}\frac{I^{\infty
          XYZ}_{Si}}{\lambda_{Si}}e^{-\frac{z}{\lambda_{Si}}}dz=I^{\infty
          XYZ}_{Si}(1-e^{-\frac{d_{Si}}{\lambda_{Si}}})
          \end{equation}
          The above results can easily be made more surface  sensitive
          by changing the detection angle to be $\theta$  degrees  off
          normal. $\lambda$ would then in the above formulas change to
          $\lambda*cos(\theta)$ as we saw in eqn 4.26

          \subsection{Growth Mechanisms}

          The availability to grow new structures on a substrate is today
          a very important field in material  science.  In  the  above
          example we assumed that the growth  of  the  Silicon  was  a
          layer by layer mechanism. Thus one layer of Si is  completed
          before the next layer is started and it is possible to  grow
          perfect macroscopic crystal structures in this manner. If  we
          look carefully at the Germanium signal as a  function  of  the
          Silicon coverage, we  shall  see  that  the  signal  actually
          consists  of  a  number  of  straight   lines   which   form
          an exponential decay. The ratio of the Germanium signal for a
          coverage of $\theta_{Si}$ less than one  monolayer  will  be
          given                  by                   \begin{equation}
          \frac{I_{Ge}(\theta_{Si})}{I_{Ge}^{\infty}}                =
          (1-\theta_{Si})+\theta_{Si}e^{-\frac{l_{Si}}{\lambda_{Si}}}
          \hspace{2cm} 0\leq \theta_{Si} \leq 1  \end{equation}  where
          l$_{Si}$ is the thickness of one monolayer of Silicon.

             The above equation  can  easily  be  generalised  to  the
          transition from the n$^{th}$ layer to the (n+1)$^{th}$ layer

          \begin{equation}
          \frac{I_{Ge}(\theta_{Si})}{I_{Ge}^{\infty}}                =
          (1-\theta_{Si})e^{-\frac{nl_{Si}}{\lambda_{Si}}}+\theta_{Si}e^{-\frac{(n+1)l_{Si}}{\lambda_{Si}}}
          \hspace{2cm} 0\leq \theta_{Si} \leq 1 \end{equation} where  $\theta_{Si}$
          is the amount of the n$^{th}$ layer covered with Silicon.

          The result of such an attenuation of the Germanium signal is
          shown in Figure 6.15.\\

          \vspace{9cm}

          \noindent {\bf Figure 6.15} The attenuation of the substrate
          signal as a  function  of  coverage.  The  different  curves
          represents different growth mechanisms.\\


          Unfortunately the growth mechanisms are in  general  not  as
          simple as we just have seen. They are  dictated  by  surface
          diffusion,  deposition  rates,  substrate  temperature,   and
          surface and interface energies of the elements involved.  It
          is beyond the scope of this review to go further into the
          various mechanisms, but we shall mention that there are
          other categories of growth.

            \begin{itemize}

          \item The layer by layer growth is called  a  {\bf  Frank-van  der
          Merwe growth}.

          \item Another possibility is that the first  layer  is  completed
          before the second layer is started, but then  the  layer  by
          layer growth stops and islands are formed. This is called  a
          {\bf Stranski-Krastanov growth} mode.

          \item Finally there is the possibility  that  islands  are  formed
          right from the beginning and that is referred to as  a  {\bf
          Volmer-Weber} growth mode.

                \end{itemize}
          It is only in the first case it is possible to describe  the
          process easily. In the other modes the unknown  distribution
          of islands and their sizes makes an analysis  very  difficult.
          The effects of the various  mechanisms  are  illustrated  in
          Figure 6.15.

          Recent experiments by using  Scanning  Tunnelling  Microscopy,
          where it is possible to  investigate  the  individual  atoms
          deposited on the surface, have shown that alloying   also
          complicates the above picture.

          \section{Scanning Auger Microscopy SAM}


          The maybe biggest advantage of AES is the availability
          to focus the primary electron beam to a  small  spot  as  we
          discussed in section  6.1.  Thus  very  small  areas  can  be
          analysed. If this is combined with the possibility  to  scan
          the electron beam over the surface it is possible to  obtain
          a lateral resolution of the surface  composition.

          Figure 6.16 shows an Auger spectrum of a Copper  grid  placed
          on top of a Silver foil. There are 25$\mu$m  between  each
          mask of the grid and the Auger spectrum was obtained  while
          the electron  beam  was  scanned  over  an  area  of  65$\times$100
          $\mu$m$^{2}$ with TV scan rate.\\




          If the  amount  of  secondary  electrons  emitted  from  the
          surface is measured as a function of the beam  position  it
          is possible to construct an image of the  surface.  This  is
          usually referred to as a secondary electron micrograph which
          can give a topologic information of the surface .  The  top
          panel of Figure 6.17 shows such a SEM picture of the  Copper
          grid.



          If now the analyser is set to measure the  intensity  in  a
          narrow window around for example the silver line at 355 eV it
          is possible to measure the amount of Silver in  the  surface
          as a function of the electron beam position. This is exactly
          what has been done in the rest of the panels in Figure 6.17
          for a number of the elements seen  in  Figure 6.16.  It  is
          easily seen that the grid consists of Copper and that  it  is
          placed on top of an Silver foil. Furthermore it is observed that
          the contamination by sulphur is  primarily  located  on  the
          Silver, whereas the  Chlorine  is  mainly  adsorbed  on  the
          Copper grid.  The  resolution  in  this  picture  is  around
          1$\mu$m.

  \newpage
          \vspace*{12cm}

         \noindent {\bf Figure 6.16} The Auger spectrum of  a  Copper
         grid mounted on a Silver foil and averaged over  an  area  of
         65$\times$100 $\mu$m.\\

             The methods can be refined  by  measuring  not  only  the
          intensity of the Auger line, but also the intensity  of  the
          background before and after the peak.  In  this  manner  the
          (Signal-Background)/Background can be used for  the  imaging
          and hereby topologic effects will be reduced.

          SAM is a rather time  consuming  process  dependent  on  the
          resolution required and the size of the  area  investigated.
          In general at  least  10  ms  is  required  for  each  point
          measured  (dependent  of  the  primary  current   and   thus
          dependent of the resolution of the electron beam) and  if  a
          high resolution is needed 512$\times$512  point  may  be  required.
          Such a picture will then take 2500 seconds  and  three  time  as
          long  if the background should be eliminated.

          The method is very useful in many  contexts,  especially  in
          applied material science.\\
          \newpage

          \vspace*{18cm}

          \noindent {\bf Figure 6.17} a)The SAM picture of a Copper grid
          mounted on a Silver foil. The first picture is  a  Secondary
          Electron Micrograph picture showing the  grid.  Then  follows
          the SAM pictures of b) Cu, c) Ag, d) S, and e) Cl. 

               \newpage

          \section{Problems}
             \begin{enumerate}


             \item Estimate from Figure 6.14 the mean free path for
          electrons 52 eV and 1147 eV through Si, respectively.

             How does these data fit into the universal curve shown in
          figure 2.6?


             Discuss how the signals in Figure 6.14 would behave
          if we did not have  Van der Merwe growth.

             \item Determine how the signal of Si  will  behave  as  a
          function of dosage when it  is  evaporated  onto  Ge  as  in
          Figure 6.14.

             Hint: Plot $ln(1-\frac{I(d)}{I_{0}\lambda})$ vs d where d
          is the thickness of the overlayer.





             \item In order to get a reasonable signal to noise ratio
          it is necessary to have an electron beam current of at least
          2 nA when doing AES or SAM. Estimate the fluency when the
          beam (10 KeV) is focused to a diameter of 1 $\mu$m or 200 ANGSTROM
          respectively. 200 ANGSTROM is the ultimate resolution that can be
          achieved with SAM in dedicated equipment. In the following
          we shall assume that the cross section for dissociation of
          an adsorbed molecule by the above electrons is $\approx$
          $1*10^{-20} m^{-2}$ and that we are measuring on a monolayer
          of molecules (5*10$^{14} cm^{-2}$). How long time can we
          allow ourselves to be measuring at the same point if we only
          allow for 10 \% of the molecules to dissociate. 


             \item Figure 6.18 shows an AES spectrum of fractured
          interface between inconel 600 plate on which a ceramic
          containing Al$_{2}$O$_{3}$ and MgO was adhered. Estimate
          the surface composition of the fracture if it is assumed
          that the surface region is homogeneous. The relevant surface
          sensitivity factors are given for peak to peak measurements.
          (Inconel 600 is an alloy containing roughly 76 \% Ni, 17  \%
          Cr and 7 \% Fe).

                       \newpage

                       \vspace*{13cm}

                       \noindent {\bf Figure 6.18}





\end{enumerate}


